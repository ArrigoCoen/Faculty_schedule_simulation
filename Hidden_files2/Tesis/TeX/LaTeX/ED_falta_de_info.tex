\subsection{Problemas de falta de información}

Encontramos diferentes páginas que tenían grupos sin información e incluso páginas sin información alguna. Para guardar la información consideramos sólo los grupos que al menos tenían: nombre de profesor, número de alumnos inscritos y horario. A continuación se muestran varios ejemplos con los diferentes casos de falta de información encontrados.


\begin{itemize}
\item[-] En la \figurename{~\ref{pagEnBlanco}} vemos un ejemplo de páginas en las cuales se tiene el nombre de la materia, pero no hay información de algún grupo: \url{http://www.fciencias.unam.mx/docencia/horarios/20081/1556/803}.

\begin{figure}[H]
\centering
\includegraphics[width=\textwidth]{FaltaInfo_A} %scale = 0.8
\caption[\textit{Página web en blanco}]{\textit{Se muestra un ejemplo de página web en blanco para la materia ``Graficación por Computadoras''. En este tipo de páginas no encontramos información de ningún grupo.}}\label{pagEnBlanco}
\end{figure}

\item[-] En la \figurename{~\ref{GpoSinInfo}} encontramos un ejemplo de páginas que no tienen información del salón: \url{http://www.fciencias.unam.mx/docencia/horarios/20081/119/4}.

\begin{figure}[H]
\centering
\includegraphics[scale = 0.5]{FaltaInfo_B} %width=\textwidth
\caption[\textit{Grupo sin información de salón}]{\textit{Se muestra un ejemplo de grupo sin información de salón. En este tipo páginas no se muestra el salón en el que se imparte la clase.}}\label{GpoSinInfo}
\end{figure}

\item[-] En la \figurename{~\ref{GpoSinAlumnos}} tenemos un ejemplo de páginas que tienen grupos sin información del número de alumnos inscritos: \url{http://www.fciencias.unam.mx/docencia/horarios/20112/119/630}.

\begin{figure}[H]
\centering
\includegraphics[scale = 0.8]{FaltaInfo_C} %width=\textwidth
\caption[\textit{Grupo sin información de alumnos}]{\textit{Se muestra un ejemplo de grupo sin información de alumnos. En este tipo páginas encontramos grupos que no tienen el número de alumnos inscritos.}}\label{GpoSinAlumnos}
\end{figure}

\item[-] En la \figurename{~\ref{SoloHorario}} vemos un ejemplo de páginas que tienen grupos sólo con el horario. No tienen nombre del profesor, salón, ayudante, número de alumnos, ni lugares disponibles: \url{http://www.fciencias.unam.mx/docencia/horarios/20091/119/841}. %\url{http://www.fciencias.unam.mx/docencia/horarios/20091/119/244}

\begin{figure}[H]
\centering
\includegraphics[scale = 1]{FaltaInfo_D} %width=\textwidth
\caption[\textit{Grupo sólo con horario}]{\textit{Se muestra un ejemplo de grupo sólo con horario. En este tipo páginas existen grupos que no tienen información del profesor o salón ni del número de alumnos inscritos. Sólo tienen la clave del grupo y el horario.}}\label{SoloHorario}
\end{figure}
\end{itemize}