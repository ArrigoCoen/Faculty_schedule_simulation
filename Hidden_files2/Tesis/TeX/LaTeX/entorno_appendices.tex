\begin{appendices}
\chapter{Materias agrupadas}\label{materias_agrupadas}

Vemos las materias que se actualizaron o cambiaron de nombre. Las negritas son los nombres y número de materia que tenían cuando el vector tenía 335 materias.


  \begin{itemize}
  \item Administración \textbf{(1)} -> Administración Actuarial (148) -> \textbf{Administración Actuarial del Riesgo} (288)
  
  \item Seminario de Inteligencia Artificial \textbf{(3)} -> \textbf{Recuperación y Búsqueda de Información en Textos} (257)
  
  \item \textbf{Seminario de Aplicaciones a las Ciencias Sociales y Administrativas (4)} -> Administración de Empresas de Software (258) -> Riesgo Tecnológico (278) -> Temas Selectos de Ingeniería de Software A (192)
  
  \item Probabilidad y Estadística \textbf{(5)} -> \textbf{Probabilidad I} (60)

\item \textbf{Mecánica Vectorial (6)} -> Cálculo Tensorial (248)

  
  \item Matemáticas Avanzadas de la Física \textbf{(12)} -> \textbf{Funciones Especiales y Transformadas Integrales} (53) -> Análisis de Fourier I (208) -> Análisis de Fourier II (231) -> Introducción a las Funciones Recursivas y Computabilidad (224)

  \item \textbf{Mecánica Analítica (13)} -> Introducción Matemática a la Mecánica Celeste (119)

  \item \textbf{Física Computacional (15)} -> Supercómputo (195)

  \item \textbf{Teoría de Gráficas (33)} -> Teoría de las Gráficas II (147)
  
  \item Graficas y Juegos \textbf{(36)} -> \textbf{Introducción a las Matemáticas Discretas} (311)
  
  \item Estadística I \textbf{(41)} -> \textbf{Inferencia Estadística} (300)
  
  \item Análisis de Redes \textbf{(44)} -> \textbf{Teoría de Redes} (152)
  
  \item Bases de Datos \textbf{(50)} -> Formación Científica I (330) -> Sistemas Manejadores de Bases de Datos (106) -> Sistemas de Bases de Datos (123) -> Grandes Bases de Datos (169) -> Fundamentos de Bases de Datos (241) -> Almacenes y Minería de Datos (269) -> \textbf{Manejo de Datos} (301) -> Programación II (51)
  
  \item \textbf{Análisis Numérico (54)} -> Análisis Numérico II (161) -> Temas Selectos de Análisis Numérico (321)
  
  \item \textbf{Seminario sobre Enseñanza de las Matemáticas I (56)} -> Seminario de Filosofía de la Ciencia I (118) -> Didáctica de las Matemáticas (319)
  
  \item Estadística II \textbf{(59)} -> \textbf{Modelos no Paramétricos y de Regresión} (284) -> Análisis de Regresión (113)
  
  \item Teoría de la Computación \textbf{(67)} -> \textbf{Autómatas y Lenguajes Formales} (240)
  
  \item Matemáticas Discretas \textbf{(68)} -> \textbf{Estructuras Discretas} (220)
  
  \item Programación I \textbf{(69)} -> \textbf{Programación} (287)
  
  \item \textbf{Procesos Estocásticos I (70)} -> Procesos Estocásticos (159)

  \item \textbf{Seminario de Geometría A (73)} -> Álgebra Geométrica (207) -> Geometría Algebraica II (209)
  
  \item Fianzas \textbf{(78)} -> Matemáticas Actuariales del Seguro de Daños (79) -> \textbf{Matemáticas Actuariales para Seguro de Daños, Fianzas y Reaseguro} (SN) -> Matemáticas Actuariales para Seguro de Daños (297) -> Reaseguro (98) -> Reaseguro Financiero (127)
  
  \item Teoría de Juegos I (143) -> \textbf{Teoría de Juegos en Economía (80)}

  \item Finanzas II \textbf{(82)} -> \textbf{Métodos Cuantitativos en Finanzas} (298)
  
  \item Seminario de Aplicaciones Actuariales I (303) -> Seminario de Matemáticas Actuariales Aplicadas \textbf{(85)} -> Seminario de Aplicaciones Actuariales II (323) -> \textbf{Seminario de Aplicaciones Actuariales} (142) -> /Seminario de Aplicaciones Actuariales I/Seminario de Estadística I (333) -> Seminario de Probabilidad A (211) -> Teoría de la Medida II (313)
  
  \item Finanzas I \textbf{(86)} -> \textbf{Mercados Financieros y Valuación de Instrumentos} (306) -> Valuación de Opciones (128)

  \item Problemas Socio-Económicos de México \textbf{(87)} -> \textbf{Análisis del México Contemporáneo} (275) -> México: Nación Multicultural (176)
  
  \item Formación Científica II \textbf{(88)} -> \textbf{Economía} (304) -> Economía I (88)

  \item Productos Financieros Derivados I \textbf{(91)} -> Productos Financieros Derivados II (184) -> \textbf{Productos Financieros Derivados} (326)
  
  \item Economía II \textbf{(93)} -> \textbf{Temas Selectos de Economía} (307) -> Econometría II (233)
  
  \item Demografía I \textbf{(94)} -> Demografía II (97) -> \textbf{Demografía} (289) -> Demografía Avanzada (308)

  \item Introducción a Ciencias de la Computación I \textbf{(95)} -> Introducción a Ciencias de la Computación II (101) -> \textbf{Introducción a Ciencias de la Computación} (222) -> Estructuras de Datos (234) -> Robótica (268)
  
  \item Arquitectura de Computadoras \textbf{(102)} -> \textbf{Organización y Arquitectura de Computadoras} (245)
  
  \item Análisis de Algoritmos I \textbf{(103)} -> Análisis de Algoritmos II (205) -> \textbf{Análisis de Algoritmos} (243)
  
  \item \textbf{Lenguajes de Programación (104)} -> Lenguajes de Programación y sus Paradigmas (247) -> Semántica y Verificación (214)
  
  \item Seminario de Ciencias de la Computación A (254) -> \textbf{Seminario de Ciencias de la Computación} (SN) -> Seminario de Ciencias de la Computación B (263) -> Seminario de Temas Selectos de Computación \textbf{(105)} -> Seminario de Aplicaciones de Cómputo (133) -> Seminario de Computación Teórica (162) -> Seminario de Aplicaciones de Cómputo II (191) -> Seminario de Sistemas para Cómputo B (217) -> Seminario de Computación Teórica II (228) -> Seminario de Sistemas para Cómputo A (164) -> Administración de Sistemas Unix/Linux (282) -> Sistemas de Información Geográfica (274) -> Métodos Formales (291)
  
  \item Principios de Computación Distribuida (190) -> Computación Concurrente (259) -> \textbf{Computación Distribuida} (252) -> \textbf{(202)}
  
  \item \textbf{Animación por Computadora} (255) -> \textbf{(203)}
  
  \item Seminario de Programación \textbf{(107)} -> \textbf{Modelado y Programación} (246) -> Diseño y Programación Orientada a Objetos (168) -> Programación Funcional y Lógica (196) -> Programación de Dispositivos Móviles (277) -> Programación Declarativa (296)
  
  \item Análisis Lógico \textbf{(108)} -> \textbf{Lógica Computacional} (244) -> Lógica Computacional II (251) -> Lógicas no Clásicas (166)
  
  \item Diseño de Sistemas Digitales \textbf{(130)} -> \textbf{Diseño de Interfaces de Usuario} (272) -> Diseño de interfaces (167)

  \item Seminario de Inteligencia Artificial II (163) -> Reconocimiento de Patrones (264) -> \textbf{Reconocimiento de Patrones y Aprendizaje Automatizado} (281) -> Seminario de Temas Selectos de Computación II \textbf{(132)} -> Computación Cuántica I (267) -> Computación Cuántica II (279) -> Sistemas Expertos (198) -> Razonamiento Automatizado (292)

  \item \textbf{Seminario Filosofía de las Matemáticas (135)} -> Seminario de Filosofía de la Ciencia II (138) -> Seminario de Filosofía de la Ciencia III (155) -> Seminario de Filosofía de la Ciencia IV (146)
  
  \item Estadística III (139) -> \textbf{Modelos de Supervivencia y de Series de Tiempo} (285)  ->  Series de Tiempo \textbf{(41)}
    
  \item \textbf{Seminario Matemáticas Aplicadas I (144)} -> Seminario de Cálculo de Formas Diferenciales (273)
  
  \item Seminario de Investigación de Operaciones \textbf{(160)} -> \textbf{Temas Selectos de Investigación de Operaciones} (305)
  
  \item Temas Selectos de Ingeniería de Software B \textbf{(165)} -> Temas Selectos de Ingeniería de Software A (192) -> Tecnologías para Desarrollos en Internet (265) -> \textbf{Ingeniería de Software II} (283) -> Patrones de Diseño de Software (294)
  
  \item Diseño de Experimentos \textbf{(177)} -> \textbf{Seminario de Estadística I} (324)

  \item \textbf{Seminario de Topología B (179)} -> Topología Diferencial II (232)

  \item \textbf{Mercadotecnia de Seguros (183)} -> Contabilidad de Seguros (290)
  
  \item \textbf{Graficación por Computadoras (186)} -> Visualización (188) -> Geometría Computacional (213) -> Visión Por Computadora (293)
  
  \item Seminario de Ciencias Computacionales \textbf{(189)} -> \textbf{Taller de Herramientas Computacionales} (312) -> Sistemas Dinámicos Computacionales I (276) -> Lingüística Computacional (227) -> Herramientas de Computación para las Ciencias (229) -> Algoritmos de Apareamiento de Cadenas (286)
  
  \item Redes Neuronales y Autómatas Celulares \textbf{(193)} -> \textbf{Redes Neuronales} (302)
  
  \item Procesos Paralelos y Distribuidos \textbf{(194)} -> \textbf{Algoritmos Paralelos} (270)
  
  \item Algoritmos Genéticos \textbf{(197)} -> \textbf{Cómputo Evolutivo} (280)
  
  \item Simulación y Control \textbf{(201)} -> \textbf{Control Estadístico de la Calidad} (210)
  
  \item Introducción a la Criptografía \textbf{(262)} -> \textbf{Criptografía y Seguridad} (271)
  
  \item \textbf{Seminario de Apoyo a la Titulación en Ciencias de la Computación} (SN) -> Seminario de Apoyo a la Titulación en Ciencias de la Computación A \textbf{(315)} -> Seminario de Apoyo a la Titulación en Ciencias de la Computación B (316)
  
  \item \textbf{Seminario de Apoyo a la Titulación en Matemáticas} (SN) -> Seminario de Apoyo a la Titulación en Matemáticas A \textbf{(310)} -> Seminario de Apoyo a la Titulación en Matemáticas B (317)
  \end{itemize}%Fin de materias con múltiples nombres



%\appendix
\chapter{Resultados útiles} \label{Apend_ResultadosUtiles}

\begin{defn} \label{EMVlambda}
\textbf{Estimador máximo verosímil de $\lambda$}

Sean $X_{1}, X_{2}, \ldots, X_{n}$ una muestra aleatoria de una población con función de densidad de probabilidad $Poisson(\lambda)$. Su función de densidad es:

\begin{equation}
f(x) = \mathrm{e}^{-\lambda} \dfrac{\lambda^{x}}{x!}
\end{equation}

\begin{eqnarray*}
\mathcal{L}(X_{1}, X_{2},\ldots, X_{n}; \lambda) &=& \displaystyle \prod_{i = 1}^{n} \left( \mathrm{e}^{-\lambda} \dfrac{\lambda^{x_{i}}}{x_{i}!} \right)\\
												 &=& \mathrm{e}^{-n \lambda} \dfrac{\lambda^{\displaystyle \sum_{i = 1}^{n} x_{i}}}{\displaystyle  \prod_{i = 1}^{n} x_{i}!}\\
\text{Sacamos ln}\\
ln \mathcal{L}(X_{1}, X_{2}, \ldots, X_{n};\lambda) &=& -n\lambda + \displaystyle \sum_{i = 1}^{n} x_{i} ln \lambda - ln \displaystyle  \prod_{i = 1}^{n} x_{i}! \\
\text{Derivamos con respecto a } \lambda\\
\dfrac{\partial}{\partial \lambda} ln \mathcal{L}(\underline{X};\lambda) &=& -n + \dfrac{\displaystyle \sum_{i = 1}^{n} x_{i}}{\lambda} \\
\text{Igualamos a cero}\\
-n + \dfrac{\displaystyle \sum_{i = 1}^{n} x_{i}}{\lambda} &=& 0 \\
\Rightarrow\\
\text{Despejamos } \lambda\\
\hat{\lambda} &=& \dfrac{\displaystyle \sum_{i = 1}^{n} x_{i}}{n} = \overline{x}\\
\text{Derivamos otra vez}\\
\dfrac{\partial^{2}}{\partial \lambda} ln \mathcal{L}(\underline{X};\lambda)  &=& - \dfrac{\displaystyle \sum_{i = 1}^{n} x_{i}}{\lambda^{2}} < 0\\
\therefore \hat{\lambda} &=& \overline{x} \,\,\,\,\,\, \text{es el estimador máximo verosímil}
\end{eqnarray*}
\end{defn}






%\appendix
\chapter{Abreviaturas} %

\begin{table}[H]
\centering
\begin{tabular}{|c|c|}
\hline 
 ABREVIATURA & SIGNIFICADO \\ 
\hline 
 AG & Algoritmo Genético \\ 
\hline 
 CdC & Ciencias de la Computación \\ 
%\hline 
% ESIME & Escuela Superior de Ingeniería Mecánica y Eléctrica\\ 
\hline 
 Facultad & Facultad de Ciencias de la UNAM \\ 
\hline 
 FES & Facultad de Estudios Superiores \\
\hline 
 ITAM & Instituto Tecnológico Autónomo de México \\ 
\hline 
 MatAp & Matemáticas Aplicadas \\ 
\hline 
 TC & Tiempo Completo \\ 
\hline 
 UNAM & Universidad Nacional Autónoma de México \\ 
\hline 
 URL & Uniform Resource Locator \\ 
\hline 
 a & b \\ 
\hline 
\end{tabular} 
\caption{\textit{Abreviaturas}}
\end{table}



\chapter{Ejemplo de asignación final} \label{Ej_AsigFinal}


Sabemos que la asignación final es el mejor elemento de la última generación. Éste elemento es la matriz que presentamos a continuación. Cabe aclarar que los datos se ordenaron con respecto a la materia (en orden alfabético) y por hora (de menor a mayor). La matriz tiene 630 grupos asignados, cada uno con materia, profesor y horario correspondiente.

\dfNmatAsigFinal



\chapter{Observaciones / Notas} %

\begin{enumerate}
	\item La matriz \verb+ mat_posibles_url+ se define con un tamaño fijo antes de correr el algoritmo para que no se demore por tener un objeto que va cambiando de tamaño, por lo que al final de haberle aplicado la función se le deben de quitar los renglones que no tienen información.

  \item La función \textit{casos\_alumnos} convierte los \textit{NA} de la columna \textit{Alumnos} de \textit{m\_grande} en ceros pero al generar \textit{m\_grande\_total} y pasarla por la función \textit{limpia\_m\_grande} se eliminan los \textit{NA} y se cambian por ceros por lo tanto no es necesaria la función \textit{casos\_alumnos}, basta pasar la columna correspondiente a \textit{Alumnos} de \textit{m\_grande\_total}.
  
  \item Cuando se hacen comparaciones se toman los valores reales y se les restan los valores simulados $(Reales - \mathbb{E}[Simulados])$

  \item Con las gráficas \textit{heatmap} se revisa si el modelo es adecuado o si se debe modificar algo. Se espera que las gráficas sean de color claro ya que nos interesa que el número de grupos y alumnos simulados se parezca al real.
  
  \item Se tienen dos tipos de matrices las cuales llamaremos \textit{m\_objetivo} y \textit{m\_definición}; las matrices \textit{m\_objetivo} son las que tienen la información que se utiliza para la asignación; las matrices \textit{m\_definición} nos sirven para dos cosas:
  
	\begin{enumerate}
		\item Respaldo de la descripción de cada columna
		
		\item Para guardar los índices en los que se encuentran las columnas
	\end{enumerate}
  
%  \item Para obtener lo índices de las columnas de las matrices tipo \textit{m\_definición} se debe buscar el nombre en la columna 1 (de \textit{m\_definición}) y se debe arrojar un número de columna el cual se toma de la columna 2 (de \textit{m\_definición}), ese valor es el que se utiliza para sacar información de \textit{m\_grande}. 
  
  \item Las matrices tipo \textit{m\_definición} son:
  
  	\begin{enumerate}
  		\item mat\_def\_columnas\_MG
		
		\item mat\_def\_grupos\_reales
		
		\item mat\_def\_grupos\_simulados
	\end{enumerate}
  
  \item Las matrices tipo \textit{m\_objetivo} son:
  
  	\begin{enumerate}
  		\item m\_grande
		
		\item m\_grande\_total
		
		\item $\ldots$
	\end{enumerate}  
  
  
  \item La función \textit{checa\_ind\_materia} se encarga de obtener lo índices de las columnas de las matrices tipo \textit{m\_definición} para poder sacar información de \textit{m\_grande} o de \textit{m\_grande\_total}.
  
  \item Para las simulaciones se utiliza la información anterior a la del semestre que se quiere simular para no tener información real dentro  de los datos para la simulación.
  
  \item En caso de querer elegir la capacidad del salón se va a elegir la mayor de sus capacidades (comparando las capacidades que se han tenido a lo largo de varios semestres).
  
  \item Las matrices \textit{m\_grande} y de \textit{m\_grande\_total} tienen información real.
  
%  \item Al actualizar los renglones de \textit{m\_grande} sólo se se registran los horarios y salones en los que los profesores imparten su clase, no se toman en cuenta las clases impartidas por los ayudantes.
%  
%  Ej. \url{http://www.fciencias.unam.mx/docencia/horarios/20181/2055/1323}
%  
%  El profesor imparte su clase los lunes, miércoles y viernes de 13-14hrs en el salón O215, hay una ayudantía los martes y jueves de 13-14hrs en el salón O215 y otra ayudantía los martes de 11-13hrs en el salón 304 (Yelizcalli).
%  
%  Se considera que esta materia inicia a las 13hrs y se imparte en el salón O215.
%  
%\begin{figure}[H]
%\centering
%\includegraphics[scale = 0.65]{Ej_gpo_horarios_multiples} %width=\textwidth
%\caption{\textit{Ejemplo de grupo con horarios múltiples}}
%\end{figure}

  \item En los ciclos que recorren renglones y columnas de matrices, siempre es más rápido hacer (de afuera hacia adentro) primero las columnas y luego los renglones.
  
  Si se tiene una matriz con entradas $(i,j)$ entonces:
  
  \begin{lstlisting}[language=R, caption= \textit{Ejemplo de ciclo for}]
	for(j){
	 for(i){
      m[i,j]
    }
   }
  \end{lstlisting}

%\item[-] Las materias de inglés no se imparten todos los días de la semana, en algunos casos se imparten clases en línea; los días en que se imparten las clases presenciales se van a colocar en la columna \textit{Horario} para que sólo haya números en la columna \textit{horario\_num}.

  \item El vector \textit{vec\_nom\_materias\_total} tiene los nombres de las materias, sin repeticiones, que se utiliza para las simulaciones.
  
  \item El vector \textit{vec\_excepciones} tiene las posibles ecxepciones en las que las funciones que extraen información pueden caer, de esta manera se pueden generar nuevas funciones para corregir esos casos.
  
  \item La siguiente imagen es el resultado de la función \textit{imprime\_info\_idiomas} la cual muestra la información de los idiomas. Dicha función arroja un vector con los semestres que requieren modificación.

\begin{figure}[H]
\centering
\includegraphics[scale = 0.65]{clases_de_ingles} %width=\textwidth
\caption{\textit{Resumen de clases de inglés antes de modificación}}
\end{figure}
  
Con esta información se decidió observar caso por caso los renglones que requieren modificación para la matriz \textit{m\_grande}


  \item Debido a la situación en la que estamos viviendo actualmente, ahora más que nunca es necesario tener un programa para la asignación de horarios que permita la realización de las asignaciones sin tener la necesidad de hacer reuniones en persona, ya que al proseguir con las medidas de distanciamiento social, las reuniones antiguamente hechas en persona se tendrían que hacer por medio de alguna plataforma digital las cuales no necesariamente son las más óptimas ya que dependen de la señal de todos los participantes para que haya una comunicación de manera fluída. Debido a ésto, el programa es una buena solución.

  \item Al hacer las simulaciones del número de alumnos el redondeo es hacia arriba, usando la función \textit{ceiling}.
  
  \item El vector \textit{vec\_nom\_materias\_total}, que contiene el nombre de las materias se definió en la lista \textit{param} para poder tomarlo en las diferentes funciones.
  
  \item Para resolver un problema, pensar en los pasos en los que se puede dividir dicho problema, usualmente se requieren entre 3 y 8 pasos o casos para obtener un producto final. Para cada paso hacer una función. 
  
  Se tienen dos posibles estructuras:
  
  \begin{itemize}
  \item[a)] La función del paso $n$ manda a llamar a la del paso $n-1$.
  
  Ej.

  $\textit{simula\_grupos} \left\{ \textit{simula\_gpos\_1\_sem}\right. \left\{ \textit{simula\_gpos\_1\_materia}\right. \left\{ \textit{simula\_tam\_gpo}\right.$\\
  
  \item[b)] Se tiene una función principal que manda a llamar a las funciones de cada paso:
  
  Ej.

%  \begin{lstlisting}[language=R, caption= Ejemplo 2 de estructura de funciones]
  \begin{lstlisting}[language=R, caption= \textit{Ejemplo de estructura de funciones}]
  gen_asignacion_completa <- function(sem_ini,sem_fin){   
    # Se carga y se limpia la lista de urls (para no tener paginas sin informacion,...)
    list_url <- Actualiza_list_url(list_url)
    
    # Se obtiene "m_grande" y se genera un archivo para cada semestre
    for(k in 1:length(semestres)){
      sem_info <- semestres[k]
      directorio_info[k] <- gen_m_grande(sem_info,list_url)
    }
    
    # Se genera el esqueleto del semestre que se quiere obtener
    mat_esqueleto <- gen_esqueleto(directorio_info,param)
        
    # Se genera la matriz de solicitudes de todos los profesores
    mat_solicitudes <- gen_solicitudes(param)
    
    # Se genera la matriz de asignaciones de todos los profesores
    mat_asignaciones <- gen_asignacion(mat_esqueleto,mat_solicitudes,param)
    
    return(mat_asignaciones)
  }
}
  \end{lstlisting}
  \end{itemize}
  
  \item Pudiera ser que haya un apéndice con ``Observaciones'' utlizando las notas escritas.
  
  \item Todo lo que se escriba debe tener un propósito, sino quitarlo.
  
  \item La información que se puede encontrar actualmente (debido a la pandemia) en las páginas web de los horarios de la Facultad no es la misma que la mostrada a lo largo del trabajo ya que ahora no se tiene información del salón, o del número de alumnos inscritos por materia, ni los lugares disponibles por grupo.
  
\begin{figure}[H]
\centering
\includegraphics[scale = 0.45]{Ej_horarios_20211} %width=\textwidth
\caption{\textit{Ejemplo de horarios de semestre 2021-1}}
%\url{http://www.fciencias.unam.mx/docencia/horarios/20211/2017/1639}
\end{figure}
  
  \item Notas de T26
\begin{figure}[H]
\centering
\includegraphics[scale = 0.4]{Notas_T26} %width=\textwidth
\caption{\textit{Notas de T26}}
\end{figure}
  
  \item En caso de tener subsecciones: entre 3 y 4
  
  \item Ser muy directa al escribir, pero explicar mucho más (platicar más). No hacer enunciados tan largos. No puede haber párrafos formados por un sólo enunciado. Escribir una idea por enunciado. No sólo escribir en párrafos, utilizar listas, tablas, ...  
  
  \item La estructura de cada párrafo debe ser de tipo \textit{reloj de arena}. Ir de lo general a lo particular y volver a lo general con una conclusión.
  
  \item Un enunciado equivale a una idea. Un párrafo equivale a un conjunto de ideas comunes.
  
  \item Sea $D = \dfrac{r - s}{s}$, donde \textit{r} son datos reales, \textit{s} datos simulados y \textit{D} la diferencia relativa, se busca que $D \in \left[ -\dfrac{1}{2},\dfrac{1}{2}\right]$. 
  
  \item Ejemplo del uso del comando \textit{Roxygen} para comentar las funciones en \textit{R}.
\begin{figure}[H]
\centering
\includegraphics[scale = 1]{Ej_Roxygen} %width=\textwidth
\caption{\textit{Ejemplo de Roxygen}}
\end{figure}

  \item Escribir en el archivo de LaTeX pequeños comentarios de la idea que se quiere transmitir en cada párrafo (de 2 a 3 palabras claves). Ésto sirve para referencias futuras y para ordenar los párrafos con mayor facilidad.
  
  \item Escribir párrafos de 2 a 3 enunciados completos, no dejar enunciados solos a menos que contengan información muy importante.
  
  \item En caso de tener más de 10 referencias bibliográficas utilizar \textit{Mendeley} para genera un archivo \textit{.bib} y  ponerlo en la tesis para tener la bibliografía.
  
  \item Cuidar el tamaño de letra en las gráficas que se pongan
  
  \item No poner abreviaturas en los títulos.
  
  \item La imagen \ref{img_en_ing_2} tiene título en inglés, se tienen 2 opciones: dejarlo así o buscar cómo cambiarlo.
  
  \item Recordar la diferencia entre:
  \begin{itemize}
  	\item[-] Número de alumnos inscritos
  	
  	\item[-] Número de alumnos reales
  	
  	\item[-] Número de alumnos que toman clase por cada horario (no se toman en cuenta los alumnos que empalman clases)
  \end{itemize}
  
  \item Para la elección de $q_{1}$ y $q_{2}$ se debe darle prioridad a la varianza no al mín y al máx porque se pueden tener casos en los que el mín y el máx estén muy cercanos a cero (gráfica superior) pero su varianza es grande. Queremos que la varianza se encuentre alrededor del cero (gráfica inferior).

\begin{figure}[H]
\centering
\includegraphics[scale = 0.3]{Ej_varianza} %width=\textwidth
\caption{\textit{Ejemplo de varianza}}
\end{figure}
  
  \item Preferir sacrificar el B/N en las imágenes impresas para tener una mejor versión digital a color.
  
  \item Guardar figuras hechas en R con el comando: \textit{dev.print(pdf, ``Figures/Fig\_Examples\_of\_GB\_distributions.pdf'',width=8, height=5)}
  
  \item Arrigo dijo que posiblemente alguien se va a quejar de no tomar en cuenta la preferencia de los profesores al realizar las solicitudes.
  
  \item Un histograma nos muestra la representación de la distribución empírica de un conjunto de datos. Cada barra en el histograma representa la frecuencia de un intervalo sobre el rango de las observaciones que se tienen.
  
  \item Cláusula 99 CCTPA: Ayuda para la impresión de la tesis.
  
  \url{https://www.personal.unam.mx/Docs/Contratos/AAPAUNAM20132015.pdf}

\begin{figure}[H]
\centering
\includegraphics[scale = 0.8]{clausula99_CCTPA} %width=\textwidth
\caption{\textit{Cláusula 99 CCTPA: Ayuda para la impresión de la tesis}}
\end{figure}
  
  \item Equivalencias de nombres para estadística:
  \begin{enumerate}
  \item Estadística I - Inferencia Estadística
  
  \item Estadística II - Modelos no Paramétricos y de Regresión
  
  \item Estadística III - Modelos de Supervivencia y de Series de Tiempo
  \end{enumerate}
  
  \item La frecuencia relativa en los histogramas no refleja directamente el porcentaje. Se debe multiplicar el valor del eje $Y$ por el ancho del intervalo por $100$ para obtener cifras en porcentaje. El área total de las barras sumará 1 (\ref{MargaritaJaimeRuthLizbeth}).
  
  \item No confundir las carpetas de \textit{Figuras} del GitHub con la del pdf.
  
  \item Ya no son necesarias las pruebas de bondad de ajuste porque los tamaños de grupo se van a simular con respecto a los profesores. Ver $T_{32} xx)$
  
  \item Los archivos \textit{README} sirven para explicar las cosas a los demás.
  
  \item Si los grupos pequeños dan muchos problemas podemos considerar quitarlos.
  
  \item Las materias que se actualizaron o cambiaron de nombre se pueden ver en \chaptername{~\ref{materias_agrupadas}}.
  
	\item Arrigo dijo que posiblemente alguien se va a quejar del hecho de que actualmente las inscripciones ya no se hacen con tira de materias firmada.
  
  \item El comando \verb+\figurename{~\ref{nom_figura}}+ imprime la palabra \textit{Figura} antes del número correspondiente a la figura de la referencia.
  
  \item El comando \verb+\chaptername{~\ref{nom_capitulo}}+ imprime la palabra \textit{Capítulo} antes del número correspondiente al capítulo de la referencia.

  \item El comando \verb+ \tablename{~\ref{nom_tabla}}+ imprime la palabra \textit{Tabla} antes del número correspondiente a la tabla de la referencia.

  \item En los 3 comandos anteriores la \verb+~+ sirve para poner un espacio entre el nombre y el número.
  
  \item Los comandos \verb+\subsecname{\ref{nom_subseccion}}+, \verb+ \secname{\ref{nom_seccion}}+, \verb+ \subsectionname{\ref{nom_subseccion}}+ y \verb+\subsectionname{\ref{nom_seccion}}+ no existen.
  
  \item Para cada figura, al momento de explicarla, pensar en el mensaje principal que se quiere transmitir y \textit{dejarla hablar} por sí sola.
  
  \item Nos interesa más el comportamiento de semestres más recientes. Darle más peso a ellos en las figuras.
  
  \item Utilizar la coma de Oxford en caso de confusión o si el último elemento es compuesto. Ej. Finanzas II, Procesos Estocásticos I, y Probabilidad y Estadística.
  
  \item Imagen que muestra el uso de Plan de estudio con sus diferentes variantes \textit{Plan de Estudio, Plan de Estudios, Planes de Estudios, Planes de Estudio}. Los links en donde se encuentran esos nombres son: \url{https://www.dgae-siae.unam.mx/educacion/planes.php} y \url{https://www.dgae.unam.mx/planes/licenciatura.html}.

\begin{figure}[H]
\centering
\includegraphics[scale = 1]{planes_de_estudio} %width=\textwidth
\caption{\textit{Nombres planes de estudio}}
\end{figure}
  
  \item No usar palabras despectivas como: restante,...
  
  \item En el modelo de mezcla de Normales tenemos el comando \verb+plot(mixmdl,which = 2)+, la opción \verb+which+ se encarga de seleccionar el tipo de gráfica que se muestra. \url{https://stackoverflow.com/questions/29044055/plot-which-parameter-where-in-r}
  
\begin{figure}[H]
\centering
\includegraphics[scale = 1]{which_in_plot} %width=\textwidth
\caption{\textit{which in plot}}
\end{figure}
  
  \item Uso de mayúsculas:
  \begin{itemize}
  \item RAE: \url{https://www.rae.es/dpd/may\%C3\%BAsculas#33c}
  
  \item Otro: \url{http://iesbinef.educa.aragon.es/lengua/ortografia/reglas/reglama.htm}
  \end{itemize}
  
  \item Uso de mayúsculas después de dos puntos RAE:
  
  \url{https://www.rae.es/dpd/dos\%20puntos}
  
  \item How to write your PhD thesis (without going insane) \url{https://www.youtube.com/watch?v=pM6orL-bGDc&ab_channel=JamesHaytonPhD}:
  
  \begin{itemize}
    \item Definir tiempos de trabajo y tiempos de trabajo.
  
  \item Ser constante. Escribir al menos una página al día.
  
  \item Escribir más de las áreas en las que se tiene mayor conocimiento que en temas que no se conocen al 100\%.
  
  \item Si se tiene un nivel de habilidad medio y el nivel del problema/reto es alto, entonces basta que uno se concentre en el problema para poder resolverlo.
  
\begin{figure}[H]
\centering
\includegraphics[scale = 0.65]{skill_vs_challenge_level} %width=\textwidth
\caption{\textit{Skill vs challenge level}}
\end{figure}  
  
  \end{itemize}
  

  
  \item El 50\% del tiempo se destina a salvar variables, comentar códigos, definir nombres correctos, hacer buenas estructuras en el código.
  
  \item Procurar aprender algo nuevo cada día (videos de 5-10min al día), como:

\begin{enumerate}
\item Ver videos de cómo hacer gráficas en R

\item \%\% > \%\% en R para filtrar información en matrices

\item Excel

\item Cosas de R
\end{enumerate}
  
  \item Tener en cuenta que el código hace cosas similares por materia. Se tienen los mismos errores por materia.
  
  \item Un programa de computadora, con que haga los mismos errores que un humano, es bueno porque su costo es menor.
  
  \item Se espera que después de la primera generación, la mejora en las calificaciones de las asignaciones sea grande. Casi es una regla para el AG.
  
  \item No poner cosas de otros si no se les menciona. Se se pone => hacer referencia/darle crédito.
  
  \item No se da 2 veces la misma materia al mismo profesor para que los alumnos tengan mayor gama de profesores para elegir.
  
  \item Mientras más información se pueda imprimir para darse cuenta de los posibles errores, es muy útil.
  
  \item $X_{4}:$ Analizar presentación: Hacer varias pruebas con distintas combinaciones y elegir el mejor estilo/presentación.
  
  \item $X_{14}:$ Revisar/Investigar al respecto del problema y resolverlo.
  
  \item 29/12/20202: Ya tenemos la herramienta, ahora hay que hacer pruebas para entenderla y explicarle a los demás su funcionamiento.
  
  \item 29/12/20202: Actualmente falta por asignar $\dfrac{1}{3}$ de los grupos. La meta es que sólo falte el $10\%$ de los grupos (o menos).
  
  \item El comando \verb@ ctrl + shift + f@, de \textit{R}, busca en todos los archivos de una carpeta.
  
  \item En una tabla:
  \begin{itemize}
  \item[-] No poner dígitos de más.
  
  \item[-] No repetir columnas.
  
  \item[-] Mientras menos renglones y columnas, el cuadro es más importante.
  \end{itemize}
  
  \item 
  
  \item 
  
  \item 
  
  \item 
  
  \item 
  
  \item 
  
  \item 
  
  \item 
  
  \item 
  
  \item 
  
  \item 
  
  \item 
\end{enumerate}


%$$\delta \alpha \eta \mathbb{I} \varepsilon \lambda:$$\\
%$$\mathbb{Q} \cup \mathbb{I} \varepsilon \mathbb{R} \varepsilon \int$$\\
%$$\int \varepsilon \mathbb{R}$$\\
%$$\mu \mathbb{I} \,\,\,\,\,\, \eta \Theta \nu \mathbb{I} \Theta ??$$

%\appendix



















\end{appendices}



%\appendix
%\chapter{Apéndices} %%


