\chapter{Simulación}

La simulación es un proceso que nos permite estudiar el comportamiento de un sistema complejo que es muy difícil de examinar de manera analítica. Nos ayuda a determinar de manera empírica las probabilidades de ciertos eventos. La simulación nos permite experimentar con diversos supuestos que podrían ser muy costosos de realizar. Las áreas en las que se utiliza la simulación como herramienta diaria son por ejemplo en biología, estadística, medicina, química, matemáticas, investigación de operaciones, física, ingeniería o en las ciencias sociales.

Algunos ejemplos de su aplicación van desde simular el lanzamiento de una moneda justa, hasta la simulación de colisiones de átomos en un acelerador de partículas. Se utiliza para todo tipo de propósitos, por ejemplo para poder realizar predicciones en base a datos históricos o enseñar a los pilotos a volar un avión sin poner en riesgo a la población al volar el avión real.

Actualmente se combinan diferentes metodologías de simulación con el software disponible, el análisis de sensibilidad y la optimización estocástica para poder obtener un mejor resultado al momento de simular sistemas que se hacen cada vez más complejos como las redes neuronales.

A lo largo de este capítulo explicaremos el proceso que seguimos para obtener la asignación de los horarios para cada materia con su respectivo profesor.

\section{Funciones hechas en R}

La función que genera esqueletos se llama \textit{gen\_esqueleto} la cual carga la función \textit{simula\_grupos} y ésta a su vez carga la función \textit{estima\_grupos}.


El diagrama de flujo de la función se puede ver en la figura \ref{DF_genAsig}.

\begin{figure}[H]
\centering
\includegraphics[scale = 0.7]{diagrama_flujo} %width=\textwidth
\caption{\textit{Diagrama de flujo de la función \textbf{gen\_asignacion}}}\label{DF_genAsig}
\end{figure}


\begin{itemize}
\item \textit{posibles\_url}

\item \textit{gen\_m\_grande}

\item \textit{gen\_m\_grande\_total}

\item \textit{gen\_esqueleto}

\item \textit{gen\_solicitudes}

\item \textit{gen\_asignacion}

\item \textit{gen\_simula\_alumnos}

\item \textit{gen\_simula\_tamano\_grupo}
\end{itemize}

\section{Obtención de los parámetros $q_{1}$ y $q_{2}$}

En esta sección vamos a explicar cómo obtuvimos los valores de $q_{1}$ y $q_{2}$. Éstos son parámetros que se introducen en la función \verb@hw()@ de \textit{R}. Dicha función corresponde al método Holt-Winters aditivo.

Los valores de $q_{1}$ y $q_{2}$ representan los cuantiles utilizados al calcular los intervalos de confianza. Por ejemplo, si $q_{1} = 80$ entonces se calcula el intervalo al $80\%$ de confianza. Si se introducen a la función los dos parámetros entonces se calculan dos intervalos, uno al $q_{1}\%$ de confianza y el otro al $q_{2}\%$ de confianza.

Primero definimos los parámetros generales necesarios para las simulaciones:
  
  \begin{enumerate}
\item Fijamos la semilla con el comando \verb@set.seed(8654)@, en \textit{R}.

\item Elegimos 3 semestres para simular la demanda del número de alumnos. Los seleccionamos de los semestres que ya teníamos guardados con información real. Hicimos una comparación entre nuestros datos simulados y los reales de cada semestre. Los semestres que elegimos fueron: 2019-1, 2019-2 y 2020-1.

\item Fijamos $k = 5$ (número de semestres que se tienen como ventana de información).

\item Fijamos $num\_sim = 10$ (número de simulaciones de la demanda de alumnos para el semestre a simular).
\end{enumerate}


Después fijamos 5 materias que consideramos representativas para hacer las pruebas iniciales: \textit{Cálculo Diferencial e Integral I}, \textit{Demografía}, \textit{Modelos no Paramétricos y de Regresión}, \textit{Administración de Riesgos Financieros} y \textit{Temas Selectos de Investigación de Operaciones}.

Tomamos 12 posibles combinaciones de valores para $q_{1}$ y $q_{2}$, las cuales podemos ver en la \tablename{~\ref{valoresQ1Q2}}. La letra \textit{L} indica que se tomó la cota inferior del intervalo al $q_{1}\%$ de confianza. La letra \textit{U} indica que se tomó la cota superior del intervalo al $q_{2}\%$ de confianza.

Con estas cotas formamos intervalos de tipo $(Lq_{1}$,$Uq_{2})$. De éstos intervalos obtuvimos el número de alumnos simulados para los 3 diferentes semestres previamente definidos y para cada una de las 5 materias elegidas.

\begin{table}[H]
\centering
\begin{tabular}{|c|c|c|c|c|}
\hline 
$\textbf{q}_{\textbf{1}} \backslash \textbf{q}_{\textbf{2}}$ & \textbf{80} & \textbf{85} & \textbf{90} & \textbf{99} \\ 
\hline 
\textbf{80} & - & L80,U85 & L80,U90 & L80,U99 \\ 
\hline 
\textbf{85} & L85,U80 & - & L85,U90 & L85,U99 \\ 
\hline 
\textbf{90} & L90,U80 & L90,U85 & - & L90,U99 \\ 
\hline 
\textbf{99} & L99,U80 & L99,U85 & L99,U90 & - \\ 
\hline 
\end{tabular} 
\caption[\textit{Posibles valores para $q_{1}$ y $q_{2}$}]{\textit{Tabla que muestra todas las combinaciones de los intervalos formados con las cotas inferiores y superiores de los intervalos de confianza al $q_{1}\%$ y al $q_{2}\%$.}}\label{valoresQ1Q2}
\end{table}


Una vez hecha la simulación obtuvimos dos matrices:

\begin{enumerate}
\item Matriz de diferencias relativas: Esta matriz se genera al restar, los datos reales menos los simulados y después dividirlos entre los reales. Ésta operación se repite para cada materia y para cada simulación.

\item Matriz con información por materia: Esta matriz tiene 6 columnas: \textit{materia, intervalo, mín, media, máx} y \textit{sd}. En el renglón $i$ se tienen los datos de la matriz de diferencias relativas de la i-ésima materia para el intervalo $(Lq_{1}$,$Uq_{2})$ correspondiente. Por ejemplo, en el primer renglón de la \figurename{~\ref{matMedDispersion}} vemos que se utilizó el intervalo $(L80,U85)$ para obtener el número de alumnos simulados para el siguiente semestre de la materia \textit{Cálculo Diferencial e Integral I}. Las columnas 3 y 5 corresponden al mínimo y al máximo error relativo de la materia mencionada. Las columnas 4 y 6 indican la media y la varianza de los errores relativos de todas las simulaciones hechas para \textit{Cálculo Diferencial e Integral I}.
\end{enumerate}

\begin{figure}[H]
\centering
\includegraphics[width=\textwidth]{mat_med_dispersion} %scale = 0.85
\caption[\textit{Matriz con información por materia}]{\textit{Se muestran los primeros 10 renglones de la tabla obtenida con información de la matriz de diferencias relativas.}}\label{matMedDispersion}
\end{figure}


Decidimos elegir $q_{1}$ y $q_{2}$ en base a la desviación estándar. A partir de la matriz con información por materia obtuvimos una matriz de dos columnas que se muestra en la \figurename{~\ref{promSD_5m_12p}}. La nueva matriz contiene en su primer columna el intervalo $(Lq_{1}$,$Uq_{2})$ correspondiente. En la segunda el promedio de la desviación estándar para cada intervalo de las 5 materias.

\begin{figure}[H]
\centering
\includegraphics[scale = 1]{prom_SD_5m_12p} %width=\textwidth
\caption[\textit{Promedio de la desviación estándar: 5 materias, 12 intervalos}]{\textit{Se muestra la tabla con el promedio de la desviación estándar para 5 materias y 12 diferentes intervalos.}}\label{promSD_5m_12p}
\end{figure}


Los datos en la \figurename{~\ref{promSD_5m_12p}} están ordenados de menor a mayor con respecto al promedio de la desviación estándar. Para la segunda prueba elegimos los primeros 6 intervalos de dicha tabla y seleccionamos otras 10 materias: \textit{Álgebra Lineal I}, \textit{Álgebra Superior II}, \textit{Cómputo Evolutivo}, \textit{Análisis Matemático IV}, \textit{Matemáticas Actuariales para Seguro de Daños, Fianzas y Reaseguro}, \textit{Análisis Numérico}, \textit{Teoría de la Medida I}, \textit{Introducción a las Matemáticas Discretas}, \textit{Inglés I} y \textit{Cálculo Diferencial e Integral IV}. La tabla con el promedio de la desviación estándar de sus datos se puede ver en la \figurename{~\ref{promSD_10m_6p}}.


\begin{figure}[H]
\centering
\includegraphics[scale = 1]{prom_SD_10m_6p} %width=\textwidth
\caption[\textit{Promedio de la desviación estándar: 10 materias, 6 intervalos}]{\textit{Se muestra la tabla con el promedio de la desviación estándar para 10 materias y 6 diferentes intervalos.}}\label{promSD_10m_6p}
\end{figure}


Para la tercera prueba elegimos, de la \figurename{~\ref{promSD_10m_6p}}, los intervalos que tuvieran un promedio en la desviación estándar menor a $0.5$. Seleccionamos otras 10 materias: \textit{Modelos de Supervivencia y de Series de Tiempo}, \textit{Teoría del Seguro}, \textit{Programación Entera}, \textit{Investigación de Operaciones}, \textit{Geometría Moderna I}, \textit{Geometría Analítica II}, \textit{Lógica Matemática I}, \textit{Cálculo Diferencial e Integral III}, \textit{Inferencia Estadística} y \textit{Manejo de Datos}. La tabla con el promedio de la desviación estándar de sus datos se puede ver en la \figurename{~\ref{promSD_10m_4p}}.


\begin{figure}[H]
\centering
\includegraphics[scale = 1]{prom_SD_10m_4p} %width=\textwidth
\caption[\textit{Promedio de la desviación estándar: 10 materias, 4 intervalos}]{\textit{Se muestra la tabla con el promedio de la desviación estándar para 10 materias y 4 diferentes intervalos.}}\label{promSD_10m_4p}
\end{figure}

Podemos ver que los valores de la \figurename{~\ref{promSD_10m_4p}} son muy parecidos entre sí. Debido a ésto, hicimos otra prueba con los mismos intervalos pero con 5 materias obligatorias y con muchos alumnos. Las materias que elegimos fueron: \textit{Geometría Analítica I}, \textit{Cálculo Diferencial e Integral II}, \textit{Mercados Financieros y Valuación de Instrumentos}, \textit{Probabilidad II} y \textit{Procesos Estocásticos I}. Hicimos la prueba para ver si había alguna diferencia en los datos y poder elegir un solo intervalo. La tabla con el promedio de la desviación estándar de sus datos se puede ver en la \figurename{~\ref{promSD_5m_4p}}.


\begin{figure}[H]
\centering
\includegraphics[scale = 1]{prom_SD_5m_4p} %width=\textwidth
\caption[\textit{Promedio de la desviación estándar: 5 materias, 4 intervalos}]{\textit{Se muestra la tabla con el promedio de la desviación estándar para 5 materias y 4 diferentes intervalos.}}\label{promSD_5m_4p}
\end{figure}

Analizando la información de las Figuras \ref{promSD_10m_4p} y \ref{promSD_5m_4p}, decidimos elegir los valores de $q_{1} = 85$ y $q_{2} = 80$. En la \figurename{~\ref{interConf}} se muestra el intervalo formado. De dicho intervalo vamos a obtener los valores para simular la demanda de alumnos del siguiente semestre, para cada materia en cada hora.

\begin{figure}[H]
\centering
\includegraphics[scale = 0.7]{intervalos_confianza} %width=\textwidth
\caption[\textit{Diagrama de los intervalos de confianza}]{\textit{Se muestra un diagrama con el intervalo del que se va a obtener el número de alumnos para la simulación de cada materia en cada hora.}}\label{interConf}
\end{figure}

Finalmente, con los valores de $q_{1} = 85$ y $q_{2} = 80$ hicimos una prueba aleatoria (eliminando la semilla). Las materias que elegimos para dicha prueba fueron: \textit{Modelos de Supervivencia y de Series de Tiempo, Teoría del Seguro, Cálculo Diferencial e Integral I,II y III, Investigación de Operaciones, Geometría Moderna I, Geometría Analítica II, Lógica Matemática I, Probabilidad I y II, Inferencia Estadística, Manejo de Datos, Matemáticas Financieras} y \textit{Procesos Estocásticos I}. En la \figurename{~\ref{mat_med_dispersion_pruebaAl}} podemos ver los resultados de la prueba aleatoria mencionada. El promedio de la desviación estándar del error relativo para todas las materias es $0.48$.%$0.4814898$.

\begin{figure}[H]
\centering
\includegraphics[width=\textwidth]{mat_med_dispersion_pruebaAleatoria} %scale = 1
\caption[\textit{Matriz con medidas de dispersión de prueba aleatoria}]{\textit{Se muestra en cada renglón la materia y el intervalo del que se tomaron los valores para la simulación. Los datos de la tabla están basados en la matriz de diferencias relativas.}}\label{mat_med_dispersion_pruebaAl}
\end{figure}
 %%Subsección %%To include subfiles in to a subfiles use \input{name_of_file}

\section{Simulación de la demanda de alumnos}

En la figura \ref{matAl_corregidos} vemos ...

\begin{figure}[H]
\centering
\includegraphics[scale = 0.8]{mat_alumnos_corregidos_EstadisticaIII} %width=\textwidth
\caption{\textit{Matriz con alumnos corregidos}}\label{matAl_corregidos}
\end{figure}

En la figura \ref{vec_alum_sim} vemos...

\begin{figure}[H]
\centering
\includegraphics[scale = 0.8]{vec_alum_sim_EstadisticaIII} %width=\textwidth
\caption{\textit{Vector con demanda simulada para el 2020-2}}\label{vec_alum_sim}
\end{figure}



\section{Simulación de tamaño de grupos} \label{SimTamGpos}

En esta sección vamos a explicar cómo hicimos la simulación del tamaño de grupos. Vamos a definir al tamaño de un grupo como el número de alumnos que va a tener cada grupo.






\section{Obtención de información para solicitudes}

Antes de iniciar las simulaciones de elección de materia y de horario obtuvimos un vector y una matriz con la información de las materias y de los profesores, respectivamente.

En el caso de las materias, el vector \textit{vec\_nom\_materias\_total} lo obtuvimos a partir de la matriz \textit{m\_grande\_total} del semestre 2008-1 al 2020-1. No tiene nombres repetidos. Tiene 333 materias.

En el caso de los profesores, la matriz \textit{mat\_nom\_prof\_total} tiene 2 columnas. En la primer columna se tienen los nombres de todos los profesores que han impartido clase desde el semestre 2015-1 hasta el 2020-1. Dichos nombres los obtuvimos de la matriz \textit{m\_grande\_total} de los semestres correspondientes.

En la segunda columna de la matriz, se tiene un $1$ si el profesor es de tiempo completo y un $0$ si no. Para llenarla ingresamos a la página \url{http://www.matematicas.unam.mx/index.php/nosotros/profesores-de-tiempo-completo} del Departamento de Matemáticas. Con la aplicación \textit{SelectorGadget} seleccionamos el vector con el nombre de los profesores de tiempo completo. En la figura \ref{profTC_SelectorGadget} podemos ver el código CSS que utilizamos para obtener los datos en R. También observamos que se seleccionaron 94 profesores.

\begin{figure}[H]
\centering
\includegraphics[scale = 0.8]{profesores_TC_SelectorGadget} %width=\textwidth
\caption{\textit{Profesores de tiempo completo: SelectorGadget}}\label{profTC_SelectorGadget}
\end{figure}

Al extraer la información en R obtuvimos un vector con 94 entradas. En la figura \ref{profTC_sinLimpiar} podemos ver los primeros 20 valores del vector. Notamos que cada entrada del vector inicia con los caracteres $\backslash n \backslash t \backslash t \backslash t \backslash t \backslash t \backslash t \backslash t$. Estos caracteres, en la presentación final de la página de internet, indican un salto de línea y las tabulaciones o espacios que se tienen de izquierda a derecha.

\begin{figure}[H]
\centering
\includegraphics[scale = 0.8]{profesores_TC_sinLimpiar} %width=\textwidth
\caption{\textit{Vector de profesores de tiempo completo}}\label{profTC_sinLimpiar}
\end{figure}

Limpiamos los datos para obtener un vector que sólo tuviera los nombres de los profesores. Eliminamos el título de cada uno porque en los horarios publicados en las páginas de la FC sus nombres no tienen título. También eliminamos los espacios finales que había en algunos nombres.

De esta manera obtuvimos el vector con el nombre de los profesores de tiempo completo del Departamento de Matemáticas. Dicho vector lo comparamos con la primer columna de la matriz \textit{mat\_nom\_prof\_total}, cuando los nombres coincidieron, pusimos un $1$ en el renglón correspondiente.

Al limpiar los datos encontramos 11 nombres que analizamos a mano porque no aparecía el $1$ en su respectivo renglón. Encontramos que no aparecía la información necesaria en la matriz \textit{mat\_nom\_prof\_total} por diferencias en los nombres. Encontramos diferencias por acentos, por mayúsculas y por nombre incompleto. En la tabla \ref{DifNomProfTC} vemos los nombres que aparecen en las páginas de FC comparados con los que aparecen en la página del Departamento de Matemáticas.

\begin{table}[h]
\centering
\resizebox{\textwidth}{!}{%
\begin{tabular}{|c|c|}
\hline 
\textbf{Nombre en páginas de FC} & \textbf{Nombre en página del Depto. de Matemáticas} \\ 
\hline 
Alejandro Ricardo Garciadiego Dantan & Alejandro Ricardo Garciadiego Dantán \\ 
\hline 
Edith Corina Sáenz Valadez & Edith Corina Sáenz Valadéz \\ 
\hline 
Emilio Esteban Lluis Puebla & Emilio Lluis Puebla \\
\hline 
Guillermo Javier Francisco Sienra Loera & Guillermo Sienra Loera \\
\hline 
María Asunción Begoña Fernández Fernández & Ma. Asunción Begoña Fernández Fernández \\ 
\hline 
María Concepción Ana Luisa Solís González-Cosío & Ana Luisa Solís González Cosío \\ 
\hline
María Isabel Puga Espinosa & Isabel Puga Espinosa \\ 
\hline 
María Lourdes Velasco Arreguí & María de Lourdes Velasco Arregui \\ 
\hline 
Mucuy-Kak del Carmen Guevara Aguirre & Mucuy-kak del Carmen Guevara Aguirre \\ 
\hline 
Oscar Alfredo Palmas Velasco & Óscar Alfredo Palmas Velasco \\ 
\hline 
Úrsula Xiomara Iturrarán Viveros & Úrsula Iturrarán Viveros \\ 
\hline 
\end{tabular}
} 
\caption{\textit{Diferencias en nombres de profesores de tiempo completo}}\label{DifNomProfTC}
\end{table}

Finalmente en la matriz \textit{mat\_nom\_prof\_total} se tiene la información de 1387 profesores de los cuales 94 son profesores de tiempo completo.

Algunas notas a considerar de esta matriz son:

\begin{itemize}
\item[-] Hay profesores que se repiten por diferencia de acentos. Ej. \textit{César Alejandro Arellano Ruíz, Luis Eduardo García Hernández}

\item[-] Hay profesores que se repiten por tener a lado el nombre de los ayudantes. Ej. \textit{Fermín Alberto Viniegra Heberlein, Edgar Vázquez Luis}

\item[-] Puede haber profesores que ya no impartan clases en la FC.
\end{itemize}


\section{Simulación de solicitudes de profesores oculta y pseudo-real}

En esta sección vamos a explicar cómo hicimos la simulación de la solicitud de los profesores. En la vida real los profesores pueden elegir libremente las materias que quieren impartir y seleccionan las horas a las que desean impartir sus clases. Dado que no contamos con esa información decidimos simular la elección de materias y horarios en base a la información que tenemos de semestres anteriores.

Como vimos en el diagrama \ref{DF_genAsig} simulamos dos veces las solicitudes de los profesores, en el proceso de asignación. A la primera vez que simulamos las solicitudes la llamaremos \textit{Solicitud oculta} y a la segunda la llamaremos \textit{Solicitud pseudo-real}. La explicación de su uso lo vemos a continuación.

\begin{itemize}
\item[-] Solicitud oculta: La llamamos oculta porque nos ayuda para la generación de los esqueletos. No influye directamente en la asignación final.

\item[-] Solicitud pseudo-real: Es la simulación de las posibles elecciones que los profesores harían en la vida real.
\end{itemize}




\subsection{Simulación de elección de materia}

\subsection{Simulación de elección de horario}



\section{Simulación de esqueletos}

Matriz de 2 columnas (Materia-Horario). En el renglón $i$ se tiene la información de cada grupo simulado para \textit{t+1}.

Se utiliza un matriz auxiliar de 3 columnas (Materia-Horario-Demanda\_Alum). En el renglón $i$ se tiene la información del número de alumnos simulados para la hora y materia correspondientes.




\section{Calificación de esqueletos de horario}

\begin{eqnarray*}
L_{materia} &=& -1 \,\,\,\,\,\,  \text{por cada materia no impartida}\\
x &=& \text{promedio}\\
y &=& \text{cupo}\\
L_{dif\_p\_c} (x,y) &=& \begin{cases}
    \dfrac{a}{190} (x-y)  & \quad \text{si } x<y\\
    - \dfrac{b}{190} (x-y)  & \quad \text{si } x\geqslant y
  \end{cases}\\
a &=& 0.5\\
b &=& 0.8\\
L_{categoria}^{1} (mat,solicitud) &=& -c(categoria - 1)\\
L_{categoria}^{2} (mat,solicitud) &=& -c_{1}(categoria)
\end{eqnarray*}

Al momento de simular las solicitudes de materias para los profesores suponemos que la que está en primer lugar es la materia que más quiere dar, en segundo lugar, la segunda que quiere dar.

Primero asignar grupos a los profesores de tiempo completo. Después asignar grupos faltantes a los profesores de asignatura.

$L_{materia}$ es la penalización por no tener en el esqueleto una materia que necesitamos.

$L_{dif\_p\_c}$ es la penalización en la asignación de salones. Se tiene un grupo de tamaño $x$ y un salón con capacidad $y$. Se penaliza con $\dfrac{a}{190}$ veces la diferencia entre $x$ y $y$ cuando el tamaño del grupo es menor a la capacidad del salón y se penaliza con $-\dfrac{b}{190}$ veces la diferencia entre $x$ y $y$ cuando el tamaño del grupo es mayor a la capacidad del salón.

El esqueleto depende de la demanda de alumnos y de las solicitudes de los profesores.

Primero se asignan materias a los profesores de tiempo completo y después a los de asignatura.

Los profesores de asignatura pueden quedarse sin materias asignadas.

Penalizaciones:

\begin{enumerate}
\item Si algún profesor pidió alguna materia y no se la dieron.

\item Si hay alumnos que necesitan una clase a alguna hora y no existe profesor que la imparta.

\item Con $\alpha \times num\_alumnos\_faltantes$ por cada alumno que te faltó en cada hora-materia que tenías que dar. $\alpha > 0$

\item Con $\beta \times num\_alumnos\_sobrantes$ por cada alumno que te pasaste en cada hora-materia que tenías que dar. $\beta > 0$

\end{enumerate}

Queremos el esqueleto con el menor valor en $\alpha + \beta$

Con esto se obtiene un nuevo esqueleto.


\section{Generación de asignaciones}

Matriz de 3 columnas (Materia-Horario-Profesor), la cual tiene la información de las asignaciones. A cada renglón de la matriz de esqueletos se agrega un profesor. Se genera con el esqueleto obtenido del proceso del AG y de las solicitudes de los profesores.

\subsection{Calificación de asignaciones de grupo}

%Los valores del modelo se muestran a continuación:

%\begin{lstlisting}[language=R, caption= \textit{Método Holt-Winters aditivo}]
%Holt-Winters' additive method 
%
%Call:
% hw(y = tsData, h = 1, seasonal = "additive", level = c(q1, q2)) 
%
%  Smoothing parameters:
%    alpha = 0.1247 
%    beta  = 1 
%    gamma = 1 
%
%  Initial states:
%    l = 111 
%    b = -22.5 
%    s = -50 50
%
%  sigma:  155.8648
%\end{lstlisting}

%\begin{figure}[H]
%\centering
%\includegraphics[scale = 0.8]{modelo_HW_total_alumnos_x_sem} %width=\textwidth
%\caption{\textit{Modelo del método aditivo de Holt-Winters: Total de alumnos por semestre}}
%\end{figure}




