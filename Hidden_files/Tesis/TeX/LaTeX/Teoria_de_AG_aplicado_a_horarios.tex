\chapter{Algoritmo Genético}

El Algoritmo Genético (AG) es un método de computación evolutiva o \textit{machine learning}, basado en la teoría sintética de la evolución. Dicha teoría, a grandes rasgos, combina el mecanismo de la selección natural de Darwin con la genética de Mendel. Nos indica que el individuo más apto sobrevive, por lo que entre mejores sean los padres, mejor es la descendencia.

Actualmente el AG se utiliza para resolver problemas de búsqueda y optimización. Las áreas de aplicación son por ejemplo: economía, finanzas, medicina, ciencias sociales, investigación de operaciones, hidráulica, aeronáutica y química. Algunos ejemplos dentro de estas áreas son: diseño de redes de agua potable, optimización de portafolios de inversión, el problema del agente viajero y asignar asientos en un evento.

Los pasos del AG son los siguientes:

\begin{enumerate}
\item Selección: Se define una población de tamaño $n$. El valor de aptitud o adaptabilidad, de cada elemento en la población, se asigna al evaluar su utilidad en la función objetivo. Entre mejor sea el elemento, más alto será su valor de adaptabilidad. Se eligen 2 elementos de la población, llamados padres. La selección de los padres depende de qué tan aptos sean. Con dichos padres se va a formar un hijo.

\item Cruce: Con cierta probabilidad se toma información de los padres. Dicha información la llamaremos genes.

\item Mutación: Cada gen agregado al hijo tiene una probabilidad pequeña de mutar.

\item Reemplazamiento: Se repiten los 3 pasos anteriores hasta formar $n$ hijos y poder reemplazar la población.
\end{enumerate}

Con este proceso se obtiene una generación. El número de generaciones así como el tamaño de la población se fijan antes de iniciar con el algoritmo. En la \figurename{~\ref{fig_AG}} podemos ver el diagrama de los pasos mencionados.

\begin{figure}[H]
\centering
\includegraphics[width=\textwidth]{diagrama_flujo_AG} %scale = 0.8
\caption[\textit{Diagrama del Algoritmo Genético}]{\textit{Se muestra el diagrama de flujo que se sigue en el Algoritmo Genético.}}\label{fig_AG}
\end{figure}

En la siguiente sección explicaremos cómo encontramos una buena asignación utilizando el AG. Cabe mencionar que Reeves y Rowe en su libro \textit{Genetic Algorithms: Principles and Perspectives} [\ref{ReevesRowe}], nos indican que se puede generar una nueva población haciendo el cruce y la mutación o utilizando sólo una de ellas. En nuestro caso, la estrategia que seguimos fue utilizar ambas.


%\section{Algoritmo Genético para obtener una buena asignación}
\section{Algoritmo Genético aplicado a los horarios}

El objetivo principal de este proyecto es obtener una matriz con la asignación final de materias, profesores y horas. Para ello haremos uso del AG. A continuación vamos a definir los términos que utilizaremos:

\begin{itemize}
\item[-] Asignación: Matriz de 3 columnas con la información de materias, profesores y horarios.

\item[-] Población: Conjunto de $n$ asignaciones.

\item[-] Padre: Asignación seleccionada, de la población, para formar un hijo.

\item[-] Gen: Vector de 3 entradas (Materia, Profesor, Horario) con la información extraída de una asignación.

\item[-] Hijo: Asignación formada a partir de los genes de 2 padres.

\item[-] Probabilidad de mutación: Debe de ser un valor pequeño. %Definimos \textit{prob\_mutacion} $ = \dfrac{1}{24} \approx 0.04$

\item[-] Generación: Se dice que se tiene una generación cuando se ha repetido $n$ veces el proceso para crear un hijo y se puede reemplazar la población.
\end{itemize}

Los pasos que seguimos para obtener la asignación son:

\begin{enumerate}
\item \textbf{Generar una población inicial}

Se generan $n$ asignaciones a partir del esqueleto simulado (ver Sección \ref{sec_esqueletos}) y de las solicitudes pseudo-reales simuladas (ver Sección \ref{SimSolicitudesProfesores}).

\item \textbf{Calificar cada asignación de la población}

Cada asignación tiene 2 tipos de calificaciones:

\begin{enumerate}
\item Por gen: Se califica cada gen de la asignación. Se premia con +5 si el profesor asignado es de tiempo completo. Se penaliza con -1 por cada asignación que pudo haber tenido un profesor de tiempo completo y tiene un profesor de asignatura. Para tener una calificación diferente para cada grupo, sumamos a cada gen una $\epsilon \in [0,0.1]$.

\item Global: Se califica la asignación completa. Se penaliza con -1 por cada grupo en el esqueleto sin profesor. Se penaliza con -10 por cada materia pedida por algún profesor de tiempo completo y no se le asignó. Se suma el promedio de las calificaciones por gen.

Nota:
Si el número máximo de asignaciones es 2 y un profesor pidió 3 o más  materias pero sólo se le asignó 1, entonces se penaliza una materia. Si se le asignaron, 2 no hay penalización.
\end{enumerate}

\item \textbf{Ordenar de acuerdo a la calificación}

Los genes de cada asignación se ordenan de menor a mayor calificación. Se quiere elegir con mayor probabilidad los genes con mejor calificación.

Las asignaciones se ordenan de menor a mayor calificación. Se desea elegir con mayor probabilidad las asignaciones con mejor calificación.

\item \textbf{Elegir 2 padres}

Los padres se eligen con probabilidad: $\mathbb{P}(\text{elegir la asignación } i \text{ ya ordenada}) = \dfrac{2i}{n(n+1)}$, donde $i$ es la posición de la asignación con respecto a su calificación.

\item \textbf{Elegir un gen}

Primero se elige al azar un padre (ambos tienen probabilidad $0.5)$. Una vez que se eligió un padre, seleccionar un gen: $\mathbb{P}(\text{elegir el gen } i \text{ ya ordenado}) = \dfrac{2i}{g(g+1)}$, donde $i$ es la posición del gen en la asignación con respecto a su calificación y $g$ es el número de genes que tiene la asignación.

\item \textbf{Mutación}

Se simula un número aleatorio, si ese número es menor a la probabilidad de mutación, entonces el gen tiene una mutación. Si un gen muta, se elige un gen de las solicitudes pseudo-reales y se intercambia por el gen previamente seleccionado.

\item \textbf{Agregar gen}

Una vez seleccionado el gen, éste se agrega al hijo.

\item \textbf{Ajustar información}

Se quita la información a los padres, del profesor en el gen elegido:

\begin{itemize}
\item[-] A esa hora y con esa materia para evitar que se elijan genes repetidos para el hijo. Esto puede ocurrir por ejemplo cuando ambos padres tienen el mismo gen.

\item[-] A esa hora porque los profesores no pueden impartir más de una clase a la misma hora.

\item[-] Con esa materia porque no se les puede asignar la misma materia en diferentes horarios.

\item[-] Cualquier materia a cualquier hora cuando el profesor ya tiene el número máximo de materias asignadas.
\end{itemize}

\item \textbf{Repetir 5 - 8}

Repetir los pasos 5 al 8 hasta que uno de los padres se quede sin genes.

\item \textbf{Añadir genes}

Agregar al hijo los genes del padre que aún tiene información.

\item \textbf{Repetir 4 - 10}

Repetir los pasos 4 al 10 $n$ veces para poder formar una generación.

\item \textbf{Reemplazar población}

Reemplazar a la población con la que se formó la generación.

\item \textbf{Repetir 2 - 12}

Repetir los pasos 2 al 12 hasta completar el número de generaciones deseadas.

\item \textbf{Asignación final}

Definir la asignación final como el hijo mejor calificado de la última generación.
\end{enumerate}


Algunas notas que se deben de considerar en la asignación final:

\begin{itemize}
\item[-] Algunos profesores se les asignaron 2 cálculos

\item[-] Hay profesores que ya no imparten clases en la Facultad (nota en la sección de los profesores)

\item[-] No se asignaron todos los grupos simulados en el esqueleto.

\item[-] Es posible que la asignación final no sea la mejor calificada de todas las generaciones.
\end{itemize}


\section{Resultados del Algoritmo Genético}

En esta sección presentamos los resultados obtenidos con el AG. Para obtener la matriz con la asignación final se simularon $6$ generaciones (población inicial más 5 reemplazamientos). El tamaño de la población para cada generación es $10$. El número de genes varía dependiendo de cada asignación.


En la \figurename{~\ref{EjcalifMejoresHijos}} vemos las calificaciones de la mejor asignación por generación. Observamos que la mejora en la calificación es considerable de la población inicial a la segunda generación. Notamos que la calificación del mejor elemento de la generación 5 fue menor al de la generación 4.

\begin{figure}[H]
\centering
\includegraphics[width=\textwidth]{calif_mejores_hijos.pdf} %scale = 0.7
\caption[\textit{Calificaciones de mejores asignaciones}]{\textit{Se muestran las calificaciones de la mejor asignación por generación. Se observa una mejora considerable en la calificación de la generación 1 a la 2.}}\label{EjcalifMejoresHijos}
\end{figure}

%En la \figurename{~\ref{EjcalifAsig}} vemos las calificaciones de las asignaciones.
%
%\begin{figure}[H]
%\centering
%\includegraphics[width=\textwidth]{ej_calif_asignaciones} %scale = 0.7
%\caption{\textit{Ejemplo con calificaciones de asignaciones.}}\label{EjcalifAsig}
%\end{figure}

En la \figurename{~\ref{EjcalifAsig_x_generacion}} vemos las calificaciones de las asignaciones por generación. Cada línea representa una generación. De abajo hacia arriba se tiene de la primer población a la sexta. Podemos ver que las calificaciones de las poblaciones 4 y 5 son muy parecidas. Al igual que en la \figurename{~\ref{EjcalifMejoresHijos}}, se puede observar una mejora considerable en las calificaciones de las asignaciones de la generación 1 a la 2.

\begin{figure}[H]
\centering
\includegraphics[width=\textwidth]{calif_asig_x_generacion.pdf} %scale = 0.7
\caption[\textit{Calificaciones de asignaciones por generación}]{\textit{Se muestran las calificaciones de las asignaciones por generación. Se observa una mejora considerable en la calificación de la generación 1 a la 2.}}\label{EjcalifAsig_x_generacion}
\end{figure}


Sabemos que la asignación final es el mejor elemento de la última generación. En la \tablename{~\ref{submatAsigFinal}} presentamos una submatriz de la asignación final. Cabe aclarar que los datos se ordenaron con respecto a la materia (en orden alfabético) y por hora (de menor a mayor). La matriz completa se puede ver el el Apéndice \ref{Ej_AsigFinal}.

\begin{table}[H]
\centering
\begin{tabular}{|c|p{7cm}|p{4.7cm}|c|}
\hline
\textbf{ } & \textbf{Materia} & \textbf{Profesor} & \textbf{Horario} \\ \hline
1 & Administración Actuarial del Riesgo & Ricardo Villegas Azcorra & 7 \\ \hline
108 & Análisis Numérico & Ursula Xiomara Iturrarán Viveros & 10 \\ \hline
109 & Análisis Numérico & Úrsula Xiomara Iturrarán Viveros & 10 \\ \hline
118 & Cálculo Diferencial e Integral I & Javier Fernández García & 7 \\ \hline
153 & Cálculo Diferencial e Integral III & Javier Fernández García & 11 \\ \hline
161 & Cálculo Diferencial e Integral IV & Héctor Méndez Lango & 10 \\ \hline
442 & Modelos de Supervivencia y de Series de Tiempo & Margarita Elvira Chávez Cano & 10 \\ \hline
444 & Modelos de Supervivencia y de Series de Tiempo & Rubén Ugalde Franco & 17 \\ \hline
445 & Modelos no Paramétricos y de Regresión & Margarita Elvira Chávez Cano & 9 \\ \hline
448 & Modelos no Paramétricos y de Regresión & Lizbeth Naranjo Albarrán & 11 \\ \hline
449 & Modelos no Paramétricos y de Regresión & Jaime Vázquez Alamilla & 12 \\ \hline
470 & Probabilidad I & Jaime Vázquez Alamilla & 8 \\ \hline
476 & Probabilidad I & Bibiana Obregón Quintana & 14 \\ \hline
496 & Procesos Estocásticos I & Sergio Iván López Ortega & 15 \\ \hline
603 & Variable Compleja I & Carisa Cano Figueroa & 13 \\ \hline
\end{tabular}
\caption[\textit{Submatriz con asignación final}]{\textit{Se muestra una submatriz de la asignación final. Cada renglón tiene la información de un grupo con una materia, profesor y horario asignado.}}\label{submatAsigFinal}
\end{table}