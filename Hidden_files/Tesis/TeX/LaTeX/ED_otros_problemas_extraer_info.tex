\subsection{Otros problemas al extraer información}

Al extraer la información surgieron otros problemas. En algunos casos se tuvieron que analizar las materias a mano para poder guardar la infomración de manera adecuada. A continuación se presentan los diferentes casos encontrados:
  
  \begin{itemize}
\item[-] Dentro de la obtención de datos del número de alumnos, no se lee la información cuando se tiene \textit{Un alumno}, ya que no se reconoce el texto \textit{Un} como el número $1$. En la \figurename{\ref{UnAlumno}} vemos un ejemplo de este caso.

\begin{figure}[H]
\centering
\includegraphics[scale = 0.8]{Ej_un_alumno} %width=\textwidth
%\url{http://www.fciencias.unam.mx/docencia/horarios/20081/119/1809}
\caption[\textit{Ejemplo de grupo con un alumno}]{\textit{Ejemplo de grupo con un alumno: En este caso se tiene el texto ``Un'' y no un número ``1''.}}\label{UnAlumno}
\end{figure}

Para resolver este problema se identificó la variable tipo \textit{string} igual a \textit{Un} para convertir la información y así poder utilizar los datos obtenidos.

\item[-] El algoritmo supone que todas las clases duran una hora y no se consideran las medias horas: \url{http://www.fciencias.unam.mx/docencia/horarios/20172/1556/820}. En la \figurename{\ref{MediasHoras}} mostramos un ejemplo en donde se considera que esa materia inicia a las 18hrs.

\begin{figure}[H]
\centering
\includegraphics[scale = 0.9]{Ej_gpo_medias_hrs} %width=\textwidth
%\url{http://www.fciencias.unam.mx/docencia/horarios/20172/1556/820}
\caption[\textit{Ejemplo de grupo con medias horas}]{\textit{Ejemplo de grupo con medias horas: Se considera que las materias inician en horas enteras y no a las medias horas.}}\label{MediasHoras}
\end{figure}

\item[-] Se tienen materias con múltiples horarios:\url{http://www.fciencias.unam.mx/docencia/horarios/20181/2055/1323}. En estos casos sólo se registran los horarios y salones en los que los profesores imparten su clase, no se toman en cuenta las clases impartidas por los ayudantes.

En la \figurename{\ref{horariosMultiples}} tenemos un ejemplo de este caso en donde el profesor imparte su clase los lunes, miércoles y viernes de 13-14hrs en el salón O215, hay una ayudantía los martes y jueves de 13-14hrs en el salón O215 y otra ayudantía los martes de 11-13hrs en el salón 304 (Yelizcalli). Se considera que esta materia inicia a las 13hrs y se imparte en el salón O215.

\begin{figure}[H]
\centering
\includegraphics[scale = 0.8]{Ej_gpo_horarios_multiples} %width=\textwidth
\caption[\textit{Ejemplo de grupo con horarios múltiples}]{\textit{Ejemplo de grupo con horarios múltiples: En estos grupos sólo se toman en cuenta los horarios y salones en los que los profesores imparten clase.}}\label{horariosMultiples}
\end{figure}

\item[-] Las materias de inglés no se imparten todos los días de la semana, en algunos casos se imparten clases en línea: \url{http://www.fciencias.unam.mx/docencia/horarios/20202/2017/1135}. Se registran únicamente los horarios de los días en que se imparten las clases presenciales. En la \figurename{\ref{casoIngles}} mostramos un ejemplo de este caso.

\begin{figure}[H]
\centering
\includegraphics[scale = 1]{Ej_gpo_ingles} %width=\textwidth
\caption[\textit{Ejemplo de grupo de inglés}]{\textit{Ejemplo de grupo de inglés: Las clases no se imparten todos los días. Hay sesiones virtuales. Sólo se toma en cuenta el horario de las clases presenciales.}}\label{casoIngles}
\end{figure}

\item[-] Se tienen grupos que no tienen la misma estructura que los tipos de grupos \textbf{A}, \textbf{B} y \textbf{C} definidos en la Sección \ref{TiposDeGpos}: \url{http://www.fciencias.unam.mx/docencia/horarios/20201/2017/872}, debido a ello el código CSS utilizado no sirve para obtener toda la información que se puede obtener del grupo. En la \figurename{\ref{GpoEstructuraDiferente}} tenemos un ejemplo de este caso en donde no se lee adecuadamente el número de alumnos inscritos en el grupo.

\begin{figure}[H]
\centering
\includegraphics[scale = 0.8]{Ej_gpo_con_estructura_diferente} %width=\textwidth
\caption[\textit{Ejemplo de grupo con estructura diferente}]{\textit{Ejemplo de grupo con estructura diferente: En estos casos no se extrae adecuadamente la infomración de los grupos porque el código CSS utilizado no corresponde a este tipo de grupos.}}\label{GpoEstructuraDiferente}
\end{figure}

\end{itemize}