\chapter{Conclusiones}

\begin{itemize}
\item[-] La división que se hizo de los datos es estadísticamente  adecuada.

\item[-] Se encontró que el AG es una buena opción para solucionar este problema de maximización.

\item[-] Este trabajo apoya las necesidades de los alumnos de la Facultad.

\item[-] El algoritmo propuesto puede ser aplicado en otras escuelas o facultades que cuenten con características similares a las de la Facultad. También puede ser utilizado para resolver problemas generales de asignación de recursos.

\item[-] En particular, para las carreras de la Facultad que no pertenecen al Departamento de Matemáticas, el programa realizado se puede ajustar. Se debe de considerar lo siguiente (cada inciso tiene un ejemplo):

\begin{itemize}
\item[a)] Las materias impartidas en los laboratorios duran 2 o más horas: \url{http://www.fciencias.unam.mx/docencia/horarios/20191/1081/830}.

\item[b)] No todas las materias se imparten todos los días: \url{http://www.fciencias.unam.mx/docencia/horarios/20192/1439/1419}.

\item[c)] Existen materias que no duran horas enteras: \url{http://www.fciencias.unam.mx/docencia/horarios/20192/1081/104}.

\item[d)] Hay materias que no inician en horas enteras: \url{http://www.fciencias.unam.mx/docencia/horarios/20172/181/1601}.
\end{itemize}
\end{itemize}
