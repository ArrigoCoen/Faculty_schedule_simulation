\chapter{Conclusiones}

\begin{enumerate}
\item Se encontró que el Algoritmo Genético es una buena opción para solucionar este problema de maximización. El problema de asignación de horarios se considera como un problema NP-duro. Por lo tanto se requiere un tiempo exponencial para encontrar la solución óptima. El Algoritmo Genético nos permite encontrar una buena solución en un periodo corto de tiempo. En la Sección \ref{Sec_Resultados_AG} se mostraron los resultados de una prueba con 8 generaciones y tamaño de población 8. El tiempo total para generar una asignación con dichos parámetros es apróximadamente 3.5 horas. Con ello se puede ayudar a agilizar la asignación de horarios actual.%no se puede resolver en un tiempo polinomial (razonable)

\item Este trabajo apoya las necesidades, con respecto a los horarios de las materias, de los alumnos de la Facultad de Ciencias. El número de grupos necesarios para el siguiente semestre se estima en base a la demanda del número de alumnos. Ésto nos permite proponer una asignación que depende del número de alumnos esperados en el siguiente semestre.

\item La división que se hizo de los datos es estadísticamente adecuada. En la Sección \ref{SimDemandaAlumnos} vemos que los datos están divididos por semestres pares e impares y por hora. El modelo distingue entre el tipo de semestre. Puede predecir el número de alumnos para un semestre par o impar. Al hacer la predicción por hora, se puede observar el comportamiento del turno matutino y el del turno vespertino.

\item Con el modelo de mezcla de normales es posible suavizar la distribución que tiene el tamaño de grupos por hora, como vimos en la Sección \ref{sec_esqueletos}. Ésto se basa en el supuesto de que los alumnos tienen la misma preferencia por una materia a cierta hora, o a la hora anterior o a la siguiente. %Es decir, si un alumno elige una materia a las 11hrs, le es indistinto si la toma a las 10hrs, a las 11hrs o a las 12hrs.

\item El poder definir grupos por medio de un archivo \textit{.xlsx} (ver Sección \ref{Sec_AG_aplicado}), otorga a los usuarios de este programa mayor control en las asignaciones. Con esto permitimos al usuario introducir parámetros iniciales que sean requeridos.

\item El algoritmo propuesto puede ser aplicado en otras escuelas o facultades que cuenten con características similares de información a las de la Facultad de Ciencias. También puede ser utilizado para resolver problemas generales de asignación de recursos.

\item Para las carreras de la Facultad de Ciencias que no pertenecen al Departamento de Matemáticas, el programa realizado se puede ajustar. Algunas extensiones pueden ser: %Se debe de considerar lo siguiente: %(cada inciso tiene un ejemplo):

%\begin{itemize}
%\item[a)] Las materias impartidas en los laboratorios duran 2 o más horas: \url{http://www.fciencias.unam.mx/docencia/horarios/20191/1081/830}.
%
%\item[b)] No todas las materias se imparten todos los días: \url{http://www.fciencias.unam.mx/docencia/horarios/20192/1439/1419}.
%
%\item[c)] Existen materias que no duran horas enteras: \url{http://www.fciencias.unam.mx/docencia/horarios/20192/1081/104}.
%
%\item[d)] Hay materias que no inician en horas enteras: \url{http://www.fciencias.unam.mx/docencia/horarios/20172/181/1601}.
%\end{itemize}

\begin{itemize}
\item[-] Las materias impartidas en los laboratorios duran 2 o más horas.

\item[-] No todas las materias se imparten todos los días.

\item[-] Existen materias que no duran horas enteras.

\item[-] Hay materias que no inician en horas enteras.
\end{itemize}

\item La aplicación \textit{SelectorGadget} de \textit{Google} es útil para automatizar el proceso de la descarga de información en páginas web. Como vimos en la Sección \ref{Sec_ED_estructura_URL}, el número de páginas de las que sacamos la información es grande. El proceso de automatización nos sirvió para extraer información en poco tiempo. Este proceso se puede aplicar a diversos problemas. %No sólo para las páginas de la Facultad de Ciencias.

\end{enumerate}

