\section{Resultados del Algoritmo Genético}

En esta sección presentamos los resultados obtenidos con el AG. Para obtener la matriz con la asignación final se simularon $6$ generaciones (población inicial más 5 reemplazamientos). El tamaño de la población para cada generación es $10$. El número de genes varía dependiendo de cada asignación.


En la \figurename{~\ref{EjcalifMejoresHijos}} vemos las calificaciones de la mejor asignación por generación. Observamos que la mejora en la calificación es considerable de la población inicial a la segunda generación. Notamos que la calificación del mejor elemento de la generación 5 fue menor al de la generación 4.

\begin{figure}[H]
\centering
\includegraphics[width=\textwidth]{calif_mejores_hijos.pdf} %scale = 0.7
\caption[\textit{Calificaciones de mejores asignaciones}]{\textit{Se muestran las calificaciones de la mejor asignación por generación. Se observa una mejora considerable en la calificación de la generación 1 a la 2.}}\label{EjcalifMejoresHijos}
\end{figure}

%En la \figurename{~\ref{EjcalifAsig}} vemos las calificaciones de las asignaciones.
%
%\begin{figure}[H]
%\centering
%\includegraphics[width=\textwidth]{ej_calif_asignaciones} %scale = 0.7
%\caption{\textit{Ejemplo con calificaciones de asignaciones.}}\label{EjcalifAsig}
%\end{figure}

%En la \figurename{~\ref{EjcalifAsig_x_generacion}} vemos las calificaciones de las asignaciones por generación. Cada línea representa una generación. De abajo hacia arriba se tiene de la primer población a la sexta. Podemos ver que las calificaciones de las poblaciones 4 y 5 son muy parecidas. Al igual que en la \figurename{~\ref{EjcalifMejoresHijos}}, se puede observar una mejora considerable en las calificaciones de las asignaciones de la generación 1 a la 2.
%
%\begin{figure}[H]
%\centering
%\includegraphics[width=\textwidth]{calif_asig_x_generacion.pdf} %scale = 0.7
%\caption[\textit{Calificaciones de asignaciones por generación}]{\textit{Se muestran las calificaciones de las asignaciones por generación. Se observa una mejora considerable en la calificación de la generación 1 a la 2.}}\label{EjcalifAsig_x_generacion}
%\end{figure}


\begin{figure}[H]
\centering
\includegraphics[width=\textwidth]{media_calif_x_generacion.pdf} %scale = 0.7
\caption[\textit{Media de calificaciones de asignaciones por generación}]{\textit{Se muestra el promedio de las calificaciones de las asignaciones por generación. Se observa una mejora considerable en la calificación de la generación 1 a la 2.}}\label{media_calif_x_generacion}
\end{figure}


\begin{figure}[H]
\centering
\includegraphics[width=\textwidth]{varianza_calif_x_generacion.pdf} %scale = 0.7
\caption[\textit{Varianza de calificaciones de asignaciones por generación}]{\textit{Se muestra la varianza de las calificaciones de las asignaciones por generación. Se observa una disminución considerable de la generación 3 a la 4.}}\label{vaar_calif_x_generacion}
\end{figure}


\begin{figure}[H]
\centering
\includegraphics[width=\textwidth]{boxplot_calif_x_generacion.pdf} %scale = 0.7
\caption[\textit{Gráfica de caja de calificaciones de asignaciones por generación}]{\textit{Se muestra la gráfica de caja de las calificaciones de las asignaciones por generación. Se observa una mejora considerable en la calificación de la generación 1 a la 2. Los puntos respresentan las calificaciones de cada asignación por generación.}}\label{boxplot_calif_x_generacion}
\end{figure}



En un esqueleto simulado para el 2020-2 se tuvieron 1018 grupos. De éstos se asignaron 612. Encontramos que 262 grupos sin asignación correspondían a materias optativas. Prácticamente $\dfrac{1}{4}$ de los grupos no asignados son de materias optativas. Ésto se debe a que se pueden asignar \textit{num\_max\_asig} materias obligatorias a los profesores por lo que ya no se les asignan las optativas. En la matriz de solicitudes puede haber grupos que no están en la asignación final. Ésto por el cruce de los padres.

\begin{eqnarray*}
\dfrac{612}{1018} &=& 60.11\% \text{grupos asignados}\\
\dfrac{262}{1018} &=& 25.73\% \text{grupos de optativas sin asignación}\\
\dfrac{1018 - (612 + 262)}{1018} &=& \dfrac{144}{1018} = 14.14\% \text{grupos de obligatorias sin asignación}
\end{eqnarray*}

\begin{figure}[H]
\centering
\includegraphics[width=\textwidth]{dif_rel_optativas_sin_gpo} %scale = 0.7
\caption[\textit{Optativas sin grupos asignados}]{\textit{Optativas sin grupos asignados}}
\end{figure}



\begin{figure}[H]
\centering
\includegraphics[width=\textwidth]{num_genes_generaciones_1_2} %scale = 0.7
\caption[\textit{Número de genes en generaciones 1 y 2}]{\textit{Número de genes en generaciones 1 y 2}}
\end{figure}



Sabemos que la asignación final es el mejor elemento de la última generación. En la \tablename{~\ref{submatAsigFinal}} presentamos una submatriz de la asignación final. Cabe aclarar que los datos se ordenaron con respecto a la materia (en orden alfabético) y por hora (de menor a mayor). La matriz completa se puede ver el el Apéndice \ref{Ej_AsigFinal}. Dicha matriz tiene 606 grupos asignados, cada uno con materia, profesor y horario correspondiente. En el semestre 2020-1 se tuvieron 747 grupos reales.


\begin{table}[H]
\centering
\begin{tabular}{|c|p{7cm}|p{4.7cm}|c|}
\hline
\textbf{ } & \textbf{Materia} & \textbf{Profesor} & \textbf{Horario} \\ \hline
1 & Administración Actuarial del Riesgo & Ricardo Villegas Azcorra & 7 \\ \hline
109 & Análisis Numérico & Úrsula Xiomara Iturrarán Viveros & 10 \\ \hline
118 & Cálculo Diferencial e Integral I & Javier Fernández García & 7 \\ \hline
153 & Cálculo Diferencial e Integral III & Javier Fernández García & 11 \\ \hline
161 & Cálculo Diferencial e Integral IV & Héctor Méndez Lango & 10 \\ \hline
442 & Modelos de Supervivencia y de Series de Tiempo & Margarita Elvira Chávez Cano & 10 \\ \hline
444 & Modelos de Supervivencia y de Series de Tiempo & Rubén Ugalde Franco & 17 \\ \hline
445 & Modelos no Paramétricos y de Regresión & Margarita Elvira Chávez Cano & 9 \\ \hline
448 & Modelos no Paramétricos y de Regresión & Lizbeth Naranjo Albarrán & 11 \\ \hline
449 & Modelos no Paramétricos y de Regresión & Jaime Vázquez Alamilla & 12 \\ \hline
470 & Probabilidad I & Jaime Vázquez Alamilla & 8 \\ \hline
476 & Probabilidad I & Bibiana Obregón Quintana & 14 \\ \hline
496 & Procesos Estocásticos I & Sergio Iván López Ortega & 15 \\ \hline
603 & Variable Compleja I & Carisa Cano Figueroa & 13 \\ \hline
\end{tabular}
\caption[\textit{Submatriz con asignación final}]{\textit{Se muestra una submatriz de la asignación final. Cada renglón tiene la información de un grupo con una materia, profesor y horario asignado.}}\label{submatAsigFinal}
\end{table}




