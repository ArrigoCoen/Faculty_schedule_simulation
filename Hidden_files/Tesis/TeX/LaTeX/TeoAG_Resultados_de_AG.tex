\section{Resultados del Algoritmo Genético}

Hicimos algunas pruebas con el AG para definir los parámetros de la asignación final. En la \tablename{~\ref{ResumenPruebasAG}} se puede ver un resumen con los resultados de dichas pruebas. A continuación explicamos lo que contienen sus columnas.

\begin{itemize}
\item[(1)] Número de generaciones, para cada prueba.

\item[(2)] Tamaño de la población, para cada generación.

\item[(3)] Calificación del mejor elemento de todas las generaciones.

\item[(4)] Número de genes de la asignación final.

\item[(5)] Calificación de la asignación final de cada prueba.
\end{itemize}

En las últimas 2 pruebas la asignación final no tuvo la mejor calificación de todas las asignaciones generadas. Ésto puede ocurrir porque el AG explora diferentes posibilidades y no necesariamente son siempre mejores que las anteriores.

%También podemos observar que las calificaciones de los mejores elementos es cada vez mayor en cada prueba. Con ello podemos concluir que al incrementar el tamaño de la población o el número de generaciones la asignación final será cada vez mejor.
Al observar la última columna notamos que al incrementar el tamaño de la población o el número de generaciones la calificación de la asignación final es cada vez mejor. Si comparamos la prueba $1$ con la $2$ vemos que hay una diferencia de $11.82$ puntos, en la calificación. Se incrementaron 10 elementos a la población de 5 a 15 elementos.

Si ahora comparamos la prueba $1$ con la $3$ notamos que la diferencia es de $65.24$ puntos, en la calificación. En este caso se incrementó el número de generaciones de 3 a 6.

Con estas observaciones concluímos que hay un mayor incremento en la calificación cuando se aumenta el número de generaciones que el tamaño de la población.

\begin{table}[H]
\centering
\begin{tabular}{|c|c|c|c|c|}
\hline 
 \textbf{Generaciones} & \textbf{Población} & \textbf{Mejor calif.} & \textbf{Genes asig. final} & \textbf{Calif. asig. final} \\ \hline
 3 & 5 & -599.43 & 677 & -599.43 \\ \hline
 3 & 15 & -587.61 & 672 & -587.61 \\ \hline
 6 & 5 & -534.19 & 679 & -534.19 \\ \hline
 6 & 10 & -527.56 & 694 & -527.56 \\ \hline
 8 & 8 & -510.52 & 682 & -510.52 \\ \hline
 10 & 5 & -521.78 & 683 & -530.79 \\ \hline
 10 & 10 & -514.41 & 682 & -515.85 \\ \hline
\end{tabular}
\caption[\textit{Resumen de las pruebas del Algoritmo Genético}]{\textit{Se muestran los datos de las pruebas del Algoritmo Genético. Se observa una mejora en la calificación cuando se incrementa el número de generaciones o el tamaño de la población.}}\label{ResumenPruebasAG}
\end{table}

%> xtable(mat_info_AG)
%% latex table generated in R 4.0.2 by xtable 1.8-4 package
%% Sun Jan 10 17:34:18 2021
%\begin{table}[ht]
%\centering
%\begin{tabular}{rrrrrrrrr}
%  \hline
% & Num\_generaciones & Tam\_pob & Tiempo & Mejor\_calif & Num\_genes\_asig\_fin & Calif\_asig\_fin & Prom\_genes\_gen1 & Prom\_genes\_generaciones \\ 
%  \hline
%1 & 3.00 & 5.00 & 39.52 & -599.43 & 677.00 & -599.43 & 658.40 & 670.50 \\ 
%  2 & 3.00 & 15.00 & 116.99 & -587.61 & 672.00 & -587.61 & 659.20 & 672.50 \\ 
%  3 & 6.00 & 5.00 & 96.63 & -534.19 & 679.00 & -534.19 & 660.20 & 675.00 \\ 
%  4 & 6.00 & 10.00 & 176.08 & -527.56 & 694.00 & -527.56 & 658.70 & 675.64 \\ 
%  5 & 10.00 & 5.00 & 156.97 & -521.78 & 683.00 & -530.79 & 654.00 & 675.20 \\ 
%  6 & 10.00 & 10.00 & 394.80 & -514.41 & 682.00 & -515.85 & 658.40 & 678.24 \\ 
%   \hline
%\end{tabular}
%\end{table}


A continuación presentaremos los resultados obtenidos con el AG correspondientes a la prueba con 8 generaciones y 8 elementos por cada población. En esta prueba obtuvimos la mejor calificación en la asignación final de todas las pruebas realizadas. El número de genes varía dependiendo de cada asignación, en este caso se encuentra entre 650 y 690.


En la \figurename{~\ref{EjcalifMejoresHijos}} vemos las calificaciones de la mejor asignación por generación. Observamos que la mejora en la calificación es considerable de la población inicial a la segunda generación. Este fenómeno es común en el AG porque la primera generación se simula de manera aleatoria. Para la segunda generación es más probable elegir los mejores elementos de la población. Incluso a nivel gen se tiene una mayor probabilidad de elegir los mejores. Es por ello que se da esa gran diferencia en las calificaciones entre la generación 1 y la 2.

Notamos que la calificación del mejor elemento de la generación 6 es menor a la calificación de la generación 5. Con ésto podemos observar que las diferentes posibilidades que explora el AG no necesariamente son siempre mejores que las anteriores.


%\begin{figure}[H]
%\centering
%\includegraphics[scale = 0.7]{calif_mejores_hijos_g08_n08_m004_U510.pdf} %width=\textwidth
%\caption[\textit{Calificaciones de las mejores asignaciones por generación}]{\textit{Se muestran las calificaciones de la mejor asignación por generación. Se observa una mejora considerable en la calificación de la generación 1 a la 2.}}\label{EjcalifMejoresHijos}
%\end{figure}

En la \figurename{~\ref{boxplot_calif_x_generacion}} observamos las gráficas de caja de las calificaciones de las asignaciones, para cada generación. Hay una mejora considerable en las calificaciones de la generación 1 a la 2. Los puntos representan las calificaciones de cada asignación por generación. Notamos que las calificaciones de las asignaciones en la generación 6, son muy parecidas. A diferencia de las calificaciones en la generación 3, las cuales varían más.

%\begin{figure}[H]
%\centering
%\includegraphics[scale = 0.7]{boxplot_calif_g08_n08_m004_U510.pdf} %width=\textwidth
%\caption[\textit{Gráficas de caja de calificaciones de asignaciones por generación}]{\textit{Se muestran las gráficas de caja de las calificaciones de las asignaciones por generación. Se observa una mejora considerable en la calificación de la generación 1 a la 2.}}\label{boxplot_calif_x_generacion}
%\end{figure}

En la \figurename{~\ref{media_calif_x_generacion}} se puede ver el promedio de las calificaciones de las asignaciones por generación. Al igual que en las Figuras \ref{EjcalifMejoresHijos} y \ref{boxplot_calif_x_generacion} se observa una mejora considerable en la calificación de la generación 1 a la 2. Podemos observar que el promedio de las calificaciones de la generación 7 es menor que el promedio de las calificaciones de la generación 6.


%\begin{figure}[H]
%\centering
%\includegraphics[scale = 0.7]{media_calif_g08_n08_m004_U510.pdf} %width=\textwidth
%\caption[\textit{Media de calificaciones de asignaciones por generación}]{\textit{Se muestra el promedio de las calificaciones de las asignaciones por generación. Se observa una mejora considerable en la calificación de la generación 1 a la 2.}}\label{media_calif_x_generacion}
%\end{figure}

En la \figurename{~\ref{var_calif_x_generacion}} se ve la varianza de las calificaciones de las asignaciones por generación. El valor más pequeño se observa en la generación 6. Al ver la \figurename{~\ref{boxplot_calif_x_generacion}} en dicha generación, notamos que los puntos tienen valores muy similares. La varianza más grande se obtiene en la generación 3. Ésto también se puede ver en la \figurename{~\ref{boxplot_calif_x_generacion}}.


\begin{figure}[H]
\centering
\includegraphics[scale = 0.67]{calif_mejores_hijos_g08_n08_m004_U510.pdf} %width=\textwidth
\caption[\textit{Calificaciones de las mejores asignaciones por generación}]{\textit{Se muestran las calificaciones de la mejor asignación por generación. Se observa una mejora considerable en la calificación de la generación 1 a la 2.}}\label{EjcalifMejoresHijos}
\end{figure}


\begin{figure}[H]
\centering
\includegraphics[scale = 0.67]{boxplot_calif_g08_n08_m004_U510.pdf} %width=\textwidth
\caption[\textit{Gráficas de caja de calificaciones de asignaciones por generación}]{\textit{Se muestran las gráficas de caja de las calificaciones de las asignaciones por generación. Se observa una mejora considerable en la calificación de la generación 1 a la 2.}}\label{boxplot_calif_x_generacion}
\end{figure}

\begin{figure}[H]
\centering
\includegraphics[scale = 0.67]{media_calif_g08_n08_m004_U510.pdf} %width=\textwidth
\caption[\textit{Media de calificaciones de asignaciones por generación}]{\textit{Se muestra el promedio de las calificaciones de las asignaciones por generación. Se observa una mejora considerable en la calificación de la generación 1 a la 2.}}\label{media_calif_x_generacion}
\end{figure}


\begin{figure}[H]
\centering
\includegraphics[scale = 0.67]{varianza_g08_n08_m004_U510.pdf} %width=\textwidth
\caption[\textit{Varianza de calificaciones de asignaciones por generación}]{\textit{Se muestra la varianza de las calificaciones de las asignaciones por generación. Se observa una disminución considerable de la generación 5 a la 6. El valor más grande se encuentra en la generación 3.}}\label{var_calif_x_generacion}
\end{figure}

En la \figurename{~\ref{boxplot_num_genes_x_generacion}} podemos ver las gráficas de caja del número de genes en las asignaciones por generación. Los puntos representan el número de genes de cada asignación por generación. Notamos que la media de la generación 2 es considerablemente mayor que la media de la generación 1.

De la generación 1 a la 6 los promedios del número de genes tienden a ser cada vez mayores. Ésto se logra al agregar los grupos que aún se pueden asignar después de hacer el cruce con los padres.

También podemos observar que el promedio del número de genes en las generaciones 7 y 8 es menor al promedio del número de genes en la generación 6. En las generaciones 4 y 7, el número de genes varía mucho.

\begin{figure}[h]
\centering
\includegraphics[scale = 0.7]{boxplot_num_genes_g08_n08_m004_U510.pdf} %width=\textwidth
\caption[\textit{Gráficas de caja del número de genes en asignaciones por generación}]{\textit{Se muestran las gráficas de caja del número de genes en asignaciones por generación. Se observa que los valores tienen una tendencia no decreciente.}}\label{boxplot_num_genes_x_generacion}
\end{figure}

%****************************************************************************

%En un esqueleto simulado para el 2020-2 se tuvieron 1018 grupos. De éstos se asignaron 612. Encontramos que 262 grupos sin asignación correspondían a materias optativas. Prácticamente $\dfrac{1}{4}$ de los grupos no asignados son de materias optativas. Ésto se debe a que se pueden asignar \textit{num\_max\_asig} materias obligatorias a los profesores por lo que ya no se les asignan las optativas. En la matriz de solicitudes puede haber grupos que no están en la asignación final. Ésto por el cruce de los padres.
%
%\begin{eqnarray*}
%\dfrac{612}{1018} &=& 60.11\% \text{grupos asignados}\\
%\dfrac{262}{1018} &=& 25.73\% \text{grupos de optativas sin asignación}\\
%\dfrac{1018 - (612 + 262)}{1018} &=& \dfrac{144}{1018} = 14.14\% \text{grupos de obligatorias sin asignación}
%\end{eqnarray*}

%\begin{figure}[H]
%\centering
%\includegraphics[width=\textwidth]{num_genes_generaciones_1_2} %scale = 0.7
%\caption[\textit{Número de genes en generaciones 1 y 2}]{\textit{Número de genes en generaciones 1 y 2}}
%\end{figure}

%Sabemos que la asignación final es el mejor elemento de la última generación. En la \tablename{~\ref{submatAsigFinal}} presentamos una submatriz de la asignación final. Cabe aclarar que los datos se ordenaron con respecto a la materia (en orden alfabético) y por hora (de menor a mayor). La matriz completa se puede ver el el Apéndice \ref{Ej_AsigFinal}. Dicha matriz tiene 630 grupos asignados, cada uno con materia, profesor y horario correspondiente. En el semestre 2020-1 se tuvieron 747 grupos reales. Se simularon 1011 grupos en \textit{mat\_esqueleto}. Las materias optativas y obligatorias de los siguientes porcentajes se consideran con respecto a la carrera de Actuaría.
%
%\begin{eqnarray*}
%\dfrac{630}{1011} &=& 62.31\% \text{grupos asignados}\\
%\dfrac{229}{1011} &=& 22.65\% \text{grupos de optativas sin asignación}\\
%\dfrac{17}{1011} &=& 1.68\% \text{grupos de inglés sin asignación}\\
%\dfrac{1011 - (630 + 229 + 17)}{1011} &=& \dfrac{135}{1011} = 13.35\% \text{grupos de obligatorias sin asignación}\\
%\dfrac{630}{747} &=& 84.33\% \text{grupos asignados comparado con el 2020-1}
%\end{eqnarray*}

Sabemos que la asignación final es el mejor elemento de la última generación. En la \tablename{~\ref{submatAsigFinal}} presentamos una submatriz de una asignación final. Cabe aclarar que los datos se ordenaron con respecto a la materia (en orden alfabético) y por hora (de menor a mayor). La matriz completa se puede ver en el Apéndice \ref{Ej_AsigFinal}.

Dicha matriz tiene 682 grupos asignados, cada uno con materia, profesor y horario correspondiente. En el semestre 2020-1 se tuvieron 747 grupos reales. Se simularon 1091 grupos en \textit{mat\_esqueleto} para el semestre 2020-2.

\begin{table}[H]
\centering
\begin{tabular}{|c|p{7cm}|p{4.7cm}|c|}
\hline
\textbf{ } & \textbf{Materia} & \textbf{Profesor} & \textbf{Horario} \\ \hline
1 & Administración Actuarial del Riesgo & Fernando Pérez Márquez & 7 \\ \hline
  133 & Cálculo Diferencial e Integral I & Elena de Oteyza de Oteyza & 7 \\ \hline
  156 & Cálculo Diferencial e Integral II & Héctor Méndez Lango & 11 \\ \hline
%  353 & Inferencia Estadística & Martha Angélica Montes Fonseca & 18 \\ \hline
  401 & Manejo de Datos & José Alfredo Cobián Campos & 13 \\ \hline
  418 & Matemáticas Financieras & Irma Rocío Zavala Sierra & 7 \\ \hline
  498 & Modelos de Supervivencia y de Series de Tiempo & Margarita Elvira Chávez Cano & 10 \\ \hline
%  505 & Modelos no Paramétricos y de Regresión & Jaime Vázquez Alamilla & 8 \\ \hline
  510 & Modelos no Paramétricos y de Regresión & Lizbeth Naranjo Albarrán & 11 \\ \hline
  534 & Probabilidad I & María Asunción Begoña Fernández Fernández & 10 \\ \hline
  539 & Probabilidad II & Daniel Alejandro Zurita Gutiérrez & 8 \\ \hline
  561 & Procesos Estocásticos I & Arrigo Coen Coria & 10 \\ \hline
  567 & Productos Financieros Derivados & Jesús Agustín Cano Garcés & 10 \\ \hline
%  679 & Variable Compleja I & Micho Durdevich Lucic & 16 \\ \hline
%  680 & Variable Compleja I & Mico Djurdjevic & 16 \\ \hline
  682 & Variable Compleja I & Jesús Manuel Mayorquín García & 19 \\ \hline
\end{tabular}
\caption[\textit{Submatriz con asignación final}]{\textit{Se muestra una submatriz de la asignación final. Cada renglón tiene la información de un grupo con una materia, profesor y horario asignado.}}\label{submatAsigFinal}
\end{table}


A continuación presentamos algunos porcentajes que nos permiten hacer diferentes comparaciones. Las materias optativas y obligatorias que se mecionan, se consideran con respecto a la carrera de Actuaría.

\begin{eqnarray*}
\dfrac{682}{1091} &=& 62.51\% \text{ grupos asignados}\\\\
\dfrac{409}{1091} &=& 37.48\% \text{ grupos no asignados}\\\\
\dfrac{241 + 18}{1091} &=& 23.73\% \text{ grupos de optativas e inglés sin asignación}\\\\
%\dfrac{18}{1091} &=& 1.64\% \text{ grupos de inglés sin asignación}\\\\
\dfrac{150}{1091} &=& 13.74\% \text{ grupos de obligatorias sin asignación}\\\\ %\dfrac{1091 - (682 + 241 + 18)}{1091}
\dfrac{682}{747} &=& 91.29\% \text{ grupos asignados comparados con el 2020-1}
%\dfrac{1091}{747} &=& 146.05\% \text{ grupos simulados para 2020-2 comparados con el 2020-1}
\end{eqnarray*}

%\begin{figure}[H]
%\centering
%\includegraphics[width=\textwidth]{dif_rel_optativas_sin_gpo} %scale = 0.7
%\caption[\textit{Optativas sin grupos asignados}]{\textit{Optativas sin grupos asignados}}
%\end{figure}

Notamos que casi una cuarta parte de los grupos simulados en \textit{mat\_esqueleto} corresponde a materias optativas y de inglés sin asignación. Ésto ocurre por dos principales motivos:

\begin{enumerate}
\item No hay solicitudes en \textit{mat\_solicitudes} de los grupos no asignados. Existen materias optativas que no se imparten todos los semestres. Por ello hay ocasiones en las que no se simula la solicitud.

\item Se les asigna otra materia a esa hora a los profesores que hayan solicitado alguna optativa. Usualmente se les asigna una obligatoria a la hora en la que solicitaron la optativa.
\end{enumerate}

La principal razón por la que se tiene alrededor de un $10\%$ de materias obligatorias sin asignar es por la forma en la que definimos \textit{mat\_esqueleto}. El modelo de mezcla de normales nos permite que los grupos se distribuyan a lo largo de las horas sin tener picos demasiado altos.

Ésto se basa en el supuesto de que los alumnos tienen la misma preferencia por una materia a cierta hora, o a la hora anterior o a la siguiente. Es decir, si un alumno elige \textit{Probabilidad I} a las 10hrs, le es indistinto si la toma a las 9hrs, a las 10hrs o a las 11hrs.

Este fenómeno se observa con mayor claridad en las materias con horarios muy específicos como \textit{Teoría del Seguro, Matemáticas Actuariales para Seguro de Daños, Fianzas y Reaseguro} o \textit{Matemáticas Actuariales del Seguro de Personas I}.

En el caso de \textit{Teoría del Seguro} se simularon 4 grupos en \textit{mat\_esqueleto}, a las 8hrs, 10hrs, 13hrs y 19hrs. Las solicitudes simuladas se pueden ver en la \figurename{~\ref{sol_TeoSeguro}}. Notamos que sólo hay solicitudes a las 8hrs y a las 19hrs. Con estos datos el algoritmo sólo puede asignar dos grupos, uno a las 8hrs y otro a las 19hrs.

\begin{figure}[H]
\centering
\includegraphics[scale = 0.8]{solicitudes_teoria_del_seguro} %width=\textwidth
\caption[\textit{Solicitudes simuladas de la materia ``Teoría del Seguro'' para el semestre 2020-2}]{\textit{Se muestran las solicitudes simuladas de la materia ``Teoría del Seguro'' para el semestre 2020-2.}}\label{sol_TeoSeguro}
\end{figure}

Por estas razones el algoritmo asigna alrededor del $60\%$ de los grupos simulados. Para aumentar este porcentaje, se pueden definir manualmente grupos de materias optativas o de inglés. Ésto en el archivo de excel definido en el paso \ref{paso_cero} del algoritmo. Dado que se elimina la información en las solicitudes antes de realizar el AG, no se tienen grupos repetidos en la asignación final.

