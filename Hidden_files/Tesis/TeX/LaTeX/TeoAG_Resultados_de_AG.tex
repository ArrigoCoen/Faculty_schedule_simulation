\section{Resultados del Algoritmo Genético}

En esta sección presentamos los resultados obtenidos con el AG. Para obtener la matriz con la asignación final se simularon $6$ generaciones (población inicial más 5 reemplazamientos). El tamaño de la población para cada generación es $10$. El número de genes varía dependiendo de cada asignación.


En la \figurename{~\ref{EjcalifMejoresHijos}} vemos las calificaciones de la mejor asignación por generación. Observamos que la mejora en la calificación es considerable de la población inicial a la segunda generación. Notamos que la calificación del mejor elemento de la generación 5 fue menor al de la generación 4.

\begin{figure}[H]
\centering
\includegraphics[width=\textwidth]{calif_mejores_hijos.pdf} %scale = 0.7
\caption[\textit{Calificaciones de mejores asignaciones}]{\textit{Se muestran las calificaciones de la mejor asignación por generación. Se observa una mejora considerable en la calificación de la generación 1 a la 2.}}\label{EjcalifMejoresHijos}
\end{figure}

%En la \figurename{~\ref{EjcalifAsig}} vemos las calificaciones de las asignaciones.
%
%\begin{figure}[H]
%\centering
%\includegraphics[width=\textwidth]{ej_calif_asignaciones} %scale = 0.7
%\caption{\textit{Ejemplo con calificaciones de asignaciones.}}\label{EjcalifAsig}
%\end{figure}

En la \figurename{~\ref{EjcalifAsig_x_generacion}} vemos las calificaciones de las asignaciones por generación. Cada línea representa una generación. De abajo hacia arriba se tiene de la primer población a la sexta. Podemos ver que las calificaciones de las poblaciones 4 y 5 son muy parecidas. Al igual que en la \figurename{~\ref{EjcalifMejoresHijos}}, se puede observar una mejora considerable en las calificaciones de las asignaciones de la generación 1 a la 2.

\begin{figure}[H]
\centering
\includegraphics[width=\textwidth]{calif_asig_x_generacion.pdf} %scale = 0.7
\caption[\textit{Calificaciones de asignaciones por generación}]{\textit{Se muestran las calificaciones de las asignaciones por generación. Se observa una mejora considerable en la calificación de la generación 1 a la 2.}}\label{EjcalifAsig_x_generacion}
\end{figure}


Sabemos que la asignación final es el mejor elemento de la última generación. Éste elemento es la matriz que presentamos a continuación. Cabe aclarar que los datos se ordenaron con respecto a la materia (en orden alfabético) y por hora (de menor a mayor). La matriz tiene 606 grupos asignados, cada uno con materia, profesor y horario correspondiente.

\dfNmatAsigFinal