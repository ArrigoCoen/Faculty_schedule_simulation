\section{Abreviaturas}

\begin{table}[H]
\centering
\begin{tabular}{|c|p{10cm}|}
\hline 
 \textbf{ABREVIATURA} & \textbf{SIGNIFICADO} \\ 
\hline 
 AG & Algoritmo Genético \\ 
\hline 
 CdC & Ciencias de la Computación \\ 
\hline 
 CSS & Cascading Style Sheets. Lenguaje de programación utilizado para crear una página de internet. \\ 
\hline 
% ESIME & Escuela Superior de Ingeniería Mecánica y Eléctrica\\ 
%\hline 
 Facultad & Facultad de Ciencias de la UNAM \\ 
%\hline 
% FES & Facultad de Estudios Superiores \\
%\hline 
% ITAM & Instituto Tecnológico Autónomo de México \\ 
\hline 
 MatAp & Matemáticas Aplicadas \\ 
\hline 
 NP & Problema que se puede resolver en un tiempo polinomial no determinístico \\ %\url{https://mathworld.wolfram.com/NP-Problem.html}
\hline 
 NP-duro & Problema tan o más difícil de resolver que un NP \\ %\url{https://mathworld.wolfram.com/NP-HardProblem.html}
 \hline 
 Página web & Página de internet \\ 
\hline 
 TC & Tiempo Completo \\ 
\hline 
 UNAM & Universidad Nacional Autónoma de México \\ 
\hline 
 URL & Uniform Resource Locator. Dirección de una página web. \\
\hline 
% a & b \\ 
%\hline 
\end{tabular} 
\caption[\textit{Abreviaturas}]{\textit{Abreviaturas utilizadas a lo largo de este trabajo.}}
\end{table}
