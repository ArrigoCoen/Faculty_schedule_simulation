\section{Obtención de nombres de materias}

Antes de iniciar las simulaciones primero obtuvimos el vector \textit{vec\_nom\_materias\_total} con información de las materias encontradas en la matriz \textit{m\_grande\_total} del semestre 2008-1 al 2020-1. Dicho vector  no tiene nombres repetidos y contiene 201 materias.

En esta sección vamos a explicar cómo obtuvimos los 201 nombres de las materias que vamos a utilizar. El motivo de obtener el vector \textit{vec\_nom\_materias\_total} antes de hacer las simulaciones es para evitar problemas como el que vimos en la \figurename{~\ref{MateriaNombresDistintos}}.


***EXPLICAR LIMPIEZA DE NOMBRES DE MATERIAS***


