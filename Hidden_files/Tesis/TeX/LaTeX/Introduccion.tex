%\chapter[Prueba]{Introducción}
\chapter*{Introducción}
\pagenumbering{arabic} %Para empezar la numeración con números naturales
\addcontentsline{toc}{chapter}{Introducción}


En este trabajo se propone un algoritmo para generar una asignación de grupos. Ésto con la finalidad de disminuir el tiempo que toma actualmente realizar los esqueletos de horarios así como las asignaciones de grupos en la Facultad de Ciencias de la UNAM (Facultad). 

Se hará un análisis estadístico del número de alumnos de la Facultad. Estos datos se recabarán de las páginas de internet (páginas web) de horarios de la Facultad. Después se obtendrá una estimación del número de alumnos por materia y por hora de las carreras de Actuaría, Ciencias de la Computacion, Matemáticas y Matemáticas Aplicadas. Dichas carreras pertenecen al Departamento de Matemáticas de la UNAM. Con esta estimación se simularán esqueletos de horarios que se calificarán de acuerdo a ciertos criterios. Se utilizará la teoría del Algoritmo Genético para resolver el problema de asignación de grupos. 

A lo largo del trabajo se utilizará el lenguaje de programación \textit{R} con el programa \textit{RStudio}. Debido a que sirve para realizar análisis estadísticos, gráficas, estimaciones, modelos matemáticos, entre otras funciones.

A continuación presentamos un pequeño resumen del contenido de cada capítulo:

En el primer capítulo se definen los conceptos y la nomenclatura que se utilizará a lo largo del trabajo. Se plantea el problema de programación lineal con su función objetivo y sus restricciones.

En el segundo capítulo se explica cómo se obtuvieron los datos de las páginas web de los horarios de la Facultad. Se muestran los problemas encontrados al limpiar la base de datos.

En el tercer capítulo se realiza un análisis estadístico del número de alumnos de la Facultad. Se dividen por semestres pares e impares, por turno y por carrera. En la última sección se hace un análisis del número de grupos por hora. Además se realizará una comparación con el número de alumnos por hora y se muestra la relación que existe entre ellos.

En el cuarto capítulo se muestra cómo se simularon las solicitudes de los profesores y la demanda de alumnos. Con dichos resultados y con el modelo de mezcla de normales se simula un esqueleto de horarios.

Considerando el esqueleto simulado y las solicitudes de profesores simuladas, en el quinto capítulo se aplica el Algoritmo Genético para obtener una buena asignación de grupos (Materia, Profesor, Hora). Se observan los datos obtenidos y se hace un análisis de ellos. %También se muestra una tabla con los tiempos registrados al generar diferentes asignaciones de horarios.


Por lo cual la lectura de esta tesis puede ser útil tanto para las personas encargadas de la realización de los horarios, como para los alumnos que quieran aprender las distintas herramientas que se utilizaron para este análisis y consolidación del algoritmo.
