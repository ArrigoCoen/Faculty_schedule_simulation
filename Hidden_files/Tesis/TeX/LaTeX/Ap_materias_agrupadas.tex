\section{Materias agrupadas} \label{materias_agrupadas}

Vemos las materias que se actualizaron o cambiaron de nombre. Las negritas son los nombres que se van a utilizar.

\begin{itemize}
\item Administración = Administración Actuarial = \textbf{Administración Actuarial del Riesgo}

\item Seminario de Inteligencia Artificial = \textbf{Recuperación y Búsqueda de Información en Textos}

\item \textbf{Seminario de Aplicaciones a las Ciencias Sociales y Administrativas} = Administración de Empresas de Software = Riesgo Tecnológico = Temas Selectos de Ingeniería de Software A

\item Probabilidad y Estadística = \textbf{Probabilidad I}

\item \textbf{Mecánica Vectorial} = Cálculo Tensorial

\item Matemáticas Avanzadas de la Física = \textbf{Funciones Especiales y Transformadas Integrales} = Análisis de Fourier I = Análisis de Fourier II = Introducción a las Funciones Recursivas y Computabilidad

\item \textbf{Mecánica Analítica} = Introducción Matemática a la Mecánica Celeste

\item \textbf{Física Computacional} = Supercómputo

\item \textbf{Teoría de Gráficas} = Teoría de las Gráficas II

\item Graficas y Juegos = \textbf{Introducción a las Matemáticas Discretas}

\item Estadística I = \textbf{Inferencia Estadística}

\item Análisis de Redes = \textbf{Teoría de Redes}

\item Bases de Datos = Formación Científica I = Sistemas Manejadores de Bases de Datos = Sistemas de Bases de Datos = Grandes Bases de Datos = Fundamentos de Bases de Datos = Almacenes y Minería de Datos = \textbf{Manejo de Datos} = Programación II

\item \textbf{Análisis Numérico} = Análisis Numérico II = Temas Selectos de Análisis Numérico

\item \textbf{Seminario sobre Enseñanza de las Matemáticas I} = Seminario de Filosofía de la Ciencia I = Didáctica de las Matemáticas

\item Estadística II = \textbf{Modelos no Paramétricos y de Regresión} = Análisis de Regresión

\item Teoría de la Computación = \textbf{Autómatas y Lenguajes Formales}

\item Matemáticas Discretas = \textbf{Estructuras Discretas}

\item Programación I = \textbf{Programación}

\item \textbf{Procesos Estocásticos I} = Procesos Estocásticos

\item \textbf{Seminario de Geometría A} = Álgebra Geométrica = Geometría Algebraica II

\item Fianzas = Matemáticas Actuariales del Seguro de Daños = \textbf{Matemáticas Actuariales para Seguro de Daños, Fianzas y Reaseguro} = Matemáticas Actuariales para Seguro de Daños = Reaseguro = Reaseguro Financiero

\item Teoría de Juegos I = \textbf{Teoría de Juegos en Economía}

\item Finanzas II = \textbf{Métodos Cuantitativos en Finanzas}

\item Seminario de Aplicaciones Actuariales I = Seminario de Matemáticas Actuariales Aplicadas = Seminario de Aplicaciones Actuariales II = \textbf{Seminario de Aplicaciones Actuariales}  = /Seminario de Aplicaciones Actuariales I/Seminario de Estadística I = Seminario de Probabilidad A = Teoría de la Medida II

\item Finanzas I = \textbf{Mercados Financieros y Valuación de Instrumentos} = Valuación de Opciones

\item Problemas Socio-Económicos de México = \textbf{Análisis del México Contemporáneo} = México: Nación Multicultural

\item Formación Científica II = \textbf{Economía} = Economía I

\item Productos Financieros Derivados I = Productos Financieros Derivados II = \textbf{Productos Financieros Derivados} 

\item Economía II = \textbf{Temas Selectos de Economía} = Econometría II 

\item Demografía I = Demografía II = \textbf{Demografía} = Demografía Avanzada

\item Introducción a Ciencias de la Computación I = Introducción a Ciencias de la Computación II = \textbf{Introducción a Ciencias de la Computación} = Estructuras de Datos = Robótica

\item Arquitectura de Computadoras = \textbf{Organización y Arquitectura de Computadoras}

\item Análisis de Algoritmos I = Análisis de Algoritmos II = \textbf{Análisis de Algoritmos}

\item \textbf{Lenguajes de Programación} = Lenguajes de Programación y sus Paradigmas = Semántica y Verificación

\item Seminario de Ciencias de la Computación A = \textbf{Seminario de Ciencias de la Computación} = Seminario de Ciencias de la Computación B = Seminario de Temas Selectos de Computación = Seminario de Aplicaciones de Cómputo = Seminario de Computación Teórica = Seminario de Aplicaciones de Cómputo II = Seminario de Sistemas para Cómputo B = Seminario de Computación Teórica II = Seminario de Sistemas para Cómputo A  = Administración de Sistemas Unix/Linux = Sistemas de Información Geográfica = Métodos Formales

\item Principios de Computación Distribuida= Computación Concurrente = \textbf{Computación Distribuida}

%\item \textbf{Animación por Computadora} (255) = \textbf{(203)}

\item Seminario de Programación = \textbf{Modelado y Programación} = Diseño y Programación Orientada a Objetos = Programación Funcional y Lógica = Programación de Dispositivos Móviles = Programación Declarativa

\item Análisis Lógico = \textbf{Lógica Computacional} = Lógica Computacional II = Lógicas no Clásicas

\item Diseño de Sistemas Digitales = \textbf{Diseño de Interfaces de Usuario} = Diseño de interfaces

\item Seminario de Inteligencia Artificial II = Reconocimiento de Patrones = \textbf{Reconocimiento de Patrones y Aprendizaje Automatizado} = Seminario de Temas Selectos de Computación II = Computación Cuántica I = Computación Cuántica II = Sistemas Expertos = Razonamiento Automatizado

\item \textbf{Seminario Filosofía de las Matemáticas} = Seminario de Filosofía de la Ciencia II = Seminario de Filosofía de la Ciencia III = Seminario de Filosofía de la Ciencia IV

\item Estadística III = \textbf{Modelos de Supervivencia y de Series de Tiempo} =  Series de Tiempo

\item \textbf{Seminario Matemáticas Aplicadas I} = Seminario de Cálculo de Formas Diferenciales

\item Seminario de Investigación de Operaciones = \textbf{Temas Selectos de Investigación de Operaciones}

\item Temas Selectos de Ingeniería de Software B = Temas Selectos de Ingeniería de Software A = Tecnologías para Desarrollos en Internet = \textbf{Ingeniería de Software II} = Patrones de Diseño de Software

\item Diseño de Experimentos = \textbf{Seminario de Estadística I}

\item \textbf{Seminario de Topología B} = Topología Diferencial II

\item \textbf{Mercadotecnia de Seguros} = Contabilidad de Seguros

\item \textbf{Graficación por Computadoras} = Visualización = Geometría Computacional = Visión Por Computadora

\item Seminario de Ciencias Computacionales = \textbf{Taller de Herramientas Computacionales} = Sistemas Dinámicos Computacionales I = Lingüística Computacional = Herramientas de Computación para las Ciencias = Algoritmos de Apareamiento de Cadenas

\item Redes Neuronales y Autómatas Celulares = \textbf{Redes Neuronales}

\item Procesos Paralelos y Distribuidos = \textbf{Algoritmos Paralelos}

\item Algoritmos Genéticos = \textbf{Cómputo Evolutivo}

\item Simulación y Control = \textbf{Control Estadístico de la Calidad}

\item Introducción a la Criptografía = \textbf{Criptografía y Seguridad}

\item \textbf{Seminario de Apoyo a la Titulación en Ciencias de la Computación} = Seminario de Apoyo a la Titulación en Ciencias de la Computación A = Seminario de Apoyo a la Titulación en Ciencias de la Computación B 

\item \textbf{Seminario de Apoyo a la Titulación en Matemáticas} = Seminario de Apoyo a la Titulación en Matemáticas A = Seminario de Apoyo a la Titulación en Matemáticas B
\end{itemize}%Fin de materias con múltiples nombres
